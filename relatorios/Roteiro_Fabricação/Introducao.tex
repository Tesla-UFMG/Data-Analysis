\chapter{Placa da GNSS}

\section{Objetivo}

Este documento visa padronizar e detalhar os processos de fabricação, montagem e validação de protótipos de placas eletrônicas, servindo como guia essencial para a equipe. O preenchimento completo e preciso de cada seção é fundamental para garantir a qualidade, rastreabilidade e cumprimento de prazos do projeto.

\section{Bloco de Alimentação}

Chicotes necessários: Nanofit 2 vias 20 cm, exposto na outra extremidade, vermelho e preto

\subsection{GLV}

\begin{table}[h]
    \centering
    \small 
    \caption{Componentes a soldar.}
    \begin{tabular}{c|c|l}
        \textbf{ID} & \textbf{Layer} & \textbf{Descrição} \\ 
        \hline
        H1 & top & Header nanofit 2 vias 90 graus para GLV \\ 
        D1 & top & Diodo retificador \\ 
        H2 & top & (somente se necessário) \\ 
        F1 & top & Fusível 2410 500 mA \\ 
        C1 & top & Cap eletrolítico 100uF PTH \\ 
        J1 & top & Jumper ON / OFF da alimentação do GLV \\ 
        R5 & top & R0805 10k  \\ 
        LED3 & top & 0805 RED \\ 
        \hline
    \end{tabular}
\end{table}

\begin{table}[h]
    \centering
    \footnotesize 
    \caption{Validação.}
    \begin{tabular}{c|c}
        \textbf{Validação} & \textbf{Descrição} \\ 
        \hline
        Entrada & Fonte de bancada configurada em 15 V / 250 mA máximo aplicada em H1 através do chicote \\ 
        Saída & 15 V em J1, LED3 aceso ao selecionar jumper \\ 
        Medição & Entre J1 e o negativo de H1 \\
        \hline
    \end{tabular}
\end{table}

\subsection{Regulador 5 V}

\begin{table}[h]
    \centering
    \small 
    \caption{Componentes a soldar.}
    \begin{tabular}{c|c|l}
        \textbf{ID} & \textbf{Layer} & \textbf{Descrição} \\ 
        \hline
        C2, C5 & bottom & C0805 1uF \\ 
        C3, C4 & bottom & C0805 100nF \\ 
        U2 & top & Regulador MP1584EN PTH \\ 
        R4  & top & R0805 10k  \\ 
        LED2  & top & 0805 RED \\ 
        J4 & top & jumper selector do 5 V \\ 
        \hline
    \end{tabular}
\end{table}

\begin{table}[h]
    \centering
    \footnotesize 
    \caption{Validação.}
    \begin{tabular}{c|c}
        \textbf{Validação} & \textbf{Descrição} \\ 
        \hline
        Entrada & Fonte de bancada configurada em 15 V / 250 mA máximo aplicada em H1 através do chicote \\ 
        Saída & 5 V em J4, LED2 aceso ao selecionar jumper \\ 
        Medição & Entre J4 (inferior) e o negativo de H1 \\
        \hline
    \end{tabular}
\end{table}

\subsection{Regulador USB-C}

\begin{table}[h]
    \centering
    \small 
    \caption{Componentes a soldar.}
    \begin{tabular}{c|c|l}
        \textbf{ID} & \textbf{Layer} & \textbf{Descrição} \\ 
        \hline
        R10, R11 & bottom & R0805 5.1k \\ 
        H5  & top & Conector USB-C \\ 
        \hline
    \end{tabular}
\end{table}

\begin{table}[h]
    \centering
    \footnotesize 
    \caption{Validação.}
    \begin{tabular}{c|c}
        \textbf{Validação} & \textbf{Descrição} \\ 
        \hline
        Entrada & Conectar o USB-C no notebook \\ 
        Saída & 5 V em J4, LED2 aceso ao selecionar jumper \\ 
        Medição & Entre J4 (superior) e o negativo de H1 \\
        \hline
    \end{tabular}
\end{table}

\subsection{Regulador 3.3 V}

\begin{table}[h]
    \centering
    \small 
    \caption{Componentes a soldar.}
    \begin{tabular}{c|c|l}
        \textbf{ID} & \textbf{Layer} & \textbf{Descrição} \\ 
        \hline
        C6 & bottom & C0805 10uF \\ 
        C7 & bottom & C0805 22uF \\ 
        U3 & bottom & Regulador 3V3 \\ 
        J3 & bottom & jumper ON / OFF da fonte 3V3 \\ 
        R1 & top & R0805 4.7k \\ 
        LED1  & top & 0805 RED \\ 
        \hline
    \end{tabular}
\end{table}

\begin{table}[h]
    \centering
    \footnotesize 
    \caption{Validação.}
    \begin{tabular}{c|c}
        \textbf{Validação} & \textbf{Descrição} \\ 
        \hline
        Entrada & Conectar o USB-C no notebook \\ 
        Saída & 3.3 V em J3, LED1 aceso ao selecionar jumper \\ 
        Medição & Entre J3 e o negativo de H1 \\
        \hline
    \end{tabular}
\end{table}

\newpage

\section{Microcontrolador (MCU)}

Chicotes necessários: Nanofit 6 vias conectado a um ST-Link V2.
Verificar pinout antes de soldar o crimpar / soldar o conector.

\subsection{Circuitos Essenciais}

\begin{table}[h]
    \centering
    \small 
    \caption{Componentes a soldar.}
    \begin{tabular}{c|c|l}
        \textbf{ID} & \textbf{Layer} & \textbf{Descrição} \\ 
        \hline
        U5 & top & STM32U5, QFP48 package \\ 
        C17, C16, C24, C20, C18 & top & C0805 100nF (acoplamento) \\ 
        C14, C15 & bottom & C0805 1uF (acoplamento) \\ 
        C28 & top & C0805 4.7uF \\ 
        XP1 & top & Cristal 8 MHz PTH  \\ 
        C13, C19 & top & C0805 22pF  \\ 
        R13 & bottom & R0805 10k  \\ 
        C22 & bottom & C0805 100nF  \\ 
        B1 & top & RESET Button  \\ 
        R12 & bottom & R0805 10k  \\ 
        C21 & bottom & C0805 100nF  \\ 
        J5 & top & BOOT Jumper  \\ 
        \hline
    \end{tabular}
\end{table}

\begin{table}[h]
    \centering
    \footnotesize 
    \caption{Validação.}
    \begin{tabular}{c|c}
        \textbf{Validação} & \textbf{Descrição} \\ 
        \hline
        Entrada & Fonte de bancada configurada em 15 V / 250 mA máximo aplicada em H1 através do chicote \\ 
        Saída & Fonte não acusa curto, dreno de cerca de 20 mA do MCU estável \\ 
        \hline
    \end{tabular}
\end{table}

\subsection{Programação e Debug}

\begin{table}[h]
    \centering
    \small 
    \caption{Componentes a soldar.}
    \begin{tabular}{c|c|l}
        \textbf{ID} & \textbf{Layer} & \textbf{Descrição} \\ 
        \hline
        H7 & top & Nanofit 6 vias 180 graus \\
        J6 & top & Jumper seletor da alimentação do ST-Link \\
        B2 & bottom & DEBUG button \\
        R15 & bottom & R0805 10k \\
        C29 & bottom & C0805 100nF \\
        R17 & top & R0805 4.7k \\
        LED7 & top & 0805 RED \\
        \hline
    \end{tabular}
\end{table}

\begin{table}[h]
    \centering
    \footnotesize 
    \caption{Validação.}
    \begin{tabular}{c|c}
        \textbf{Validação} & \textbf{Descrição} \\ 
        \hline
        Entrada & Conectar USB-C ao notebook \\ 
        Firmware & Piscar LED7 quando B2 é acionado \\ 
        Saída & LED7 pisca ao acionar B2 \\ 
        \hline
        Entrada & Conectar USB-C ao notebook \\ 
        Firmware & Piscar LED7 quando um timer dispara, com breakpoint em debug na ISR \\ 
        Saída & LED7 pisca ao disparar timer, com breakpoint na ISR \\ 
        \hline
        Entrada & Conectar USB-C ao notebook \\ 
        Firmware & Escrever Hello World na UART conectada ao PC, exibir no terminal Putty \\ 
        Saída & Hello World aparece no terminal Putty do PC \\ 
        \hline
    \end{tabular}
\end{table}

\newpage

\section{Bloco da CAN}