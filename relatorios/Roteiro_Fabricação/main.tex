\documentclass[12pt,a4paper, oneside, chapter = TITLE, hidelinks]{abntex2}
\usepackage[left=2cm,top=2cm,right=2cm,bottom=2cm]{geometry}
\usepackage[brazil]{babel} %lingua do texto
\usepackage[utf8]{inputenc} %permite caracteres como acentos e cedilhas
\usepackage{times} %usa a fonte times 

\setlength{\parindent}{1.25cm} %identação do parágrafo
\usepackage{indentfirst} % identa primeiro parágrafo 
\usepackage{graphicx} %incluir figuras
\usepackage{lastpage} %para obter o número da última página   
\pagenumbering{arabic} %Numeração de página
\usepackage{longtable}
\usepackage[dvipsnames]{xcolor}
\usepackage{colortbl}
\usepackage{fancyhdr}
\usepackage{float}
\usepackage{pdfpages}
\usepackage{multirow}
\usepackage{url}

% Código 
\usepackage{listings}
\usepackage{xcolor}
\lstset{
  language=C++,
  basicstyle=\ttfamily\footnotesize,  % Fonte monoespaçada e menor
  numbers=left,                       % Números de linha à esquerda
  numberstyle=\tiny\color{gray},     % Estilo dos números de linha
  stepnumber=1,                       % Numera todas as linhas
  numbersep=10pt,                     % Espaço entre código e números
  backgroundcolor=\color{white},     % Fundo branco
  keywordstyle=\color{blue},         % Palavras-chave em azul
  commentstyle=\color{gray},         % Comentários em cinza
  stringstyle=\color{red},           % Strings em vermelho
  breaklines=true,                   % Quebra de linha automática
  frame=single                       % Caixa ao redor do código
}


% Corrigir títulos
\addto \captionsbrazil {\renewcommand{\listfigurename}{\normalsize LISTA DE ILUSTRAÇÕES}}
\addto \captionsbrazil {\renewcommand{\listtablename}{\normalsize LISTA DE TABELAS}}
\renewcommand{\listadesiglasname}{\normalsize LISTA DE ABREVIATURAS E SIGLAS}
\addto \captionsbrazil {\renewcommand{\contentsname}{\normalsize SUMÁRIO}}

\renewcommand{\ABNTEXchapterfontsize}{\normalfont\normalsize} 
\renewcommand{\cftchapterfont}{\normalfont}

\renewcommand{\ABNTEXsectionfontsize}{\normalsize\normalfont}
\renewcommand{\cftsectionfont}{\normalfont} 

\renewcommand{\ABNTEXsubsectionfontsize}{\normalfont\normalsize}
\renewcommand{\cftsubsectionfont}{\normalfont} 

\renewcommand{\ABNTEXsubsubsectionfontsize}{\normalfont\normalsize}
\renewcommand{\cftsubsubsectionfont}{\normalfont} 

\newcommand{\un}[1]{\;\textrm{#1}}


% Múltiplos Autores
\makeatletter
\renewcommand{\imprimirautor}{
    \begin{tabular}{l}
    \@author
    \end{tabular}
}
\let\ps@abntexheadings\ps@plain % Carrega o estilo básico
\makeatother

% Cor padrão para as linhas
\definecolor{Tesla}{HTML}{08d45c}

\setlength{\arrayrulewidth}{1pt}  
\fancypagestyle{abntexheadings}{
  \fancyhf{} % Limpa todos os cabeçalhos e rodapés
  \fancyhead[R]{\thepage} % Número da página
  \fancyhead[L]{\nouppercase{\leftmark}} % Capítulo no lado esquerdo  
  \renewcommand{\headrulewidth}{1pt} % Espessura da linha
  \renewcommand{\headrule}{\color{Tesla}\hrule width\headwidth height\headrulewidth} % Linha colorida
   \renewcommand{\footnoterule}{\begingroup \color{Tesla}\hrule width\headwidth height\headrulewidth \endgroup}
  \fancyfoot[C]{} % Rodapé central vazio
  \setlength{\headheight}{1cm}
}
% Nova capa
\renewcommand{\imprimircapa}{
	\begin{capa}
            {\centering
            \begin{tabular}{c c c}
                \begin{tabular}{c}
                    \includegraphics[width=0.15\textwidth]{logos/logo_tesla.png}\\ \vspace{2.3cm}
                \end{tabular} 
                & 
                \begin{tabular}{c}
                    {\large UNIVERSIDADE FEDERAL DE MINAS GERAIS}\\
                    {\large FÓRMULA TESLA}  \vspace{2cm}
                \end{tabular} & 
               \includegraphics[width=0.12\textwidth]{logos/logo_engenharia.png}
            \end{tabular}
            \vspace{3cm}\\
		{\Large\imprimirtitulo}\\\vspace{1cm}
            {\large\imprimirpreambulo}\\}
            \vfill
            {\imprimirautor}\\
            \vfill
		{\center {\normalsize\imprimirlocal}\\
		{\normalsize\imprimirdata}\\}
		
	\end{capa}
}


\title{Relatório: GNSS}
\date{19/12/2025}
\local{Belo Horizonte}
\preambulo{Primeira Versão} % Versão
\author{
    \textit{\large Autores:}\\
    Raphael Henrique Braga Leivas
}

\begin{document}

	% \tableofcontents*  \thispagestyle{empty} % Retira a numeração da página
    
        \mainmatter % Inicia a numeração das páginas
        \pagestyle{abntexheadings}
        \chapter{Placa da GNSS}

\section{Objetivo}

Este documento visa padronizar e detalhar os processos de fabricação, montagem e validação de protótipos de placas eletrônicas, servindo como guia essencial para a equipe. O preenchimento completo e preciso de cada seção é fundamental para garantir a qualidade, rastreabilidade e cumprimento de prazos do projeto.

\section{Bloco de Alimentação}

Chicotes necessários: Nanofit 2 vias 20 cm, exposto na outra extremidade, vermelho e preto

\subsection{GLV}

\begin{table}[h]
    \centering
    \small 
    \caption{Componentes a soldar.}
    \begin{tabular}{c|c|l}
        \textbf{ID} & \textbf{Layer} & \textbf{Descrição} \\ 
        \hline
        H1 & top & Header nanofit 2 vias 90 graus para GLV \\ 
        D1 & top & Diodo retificador \\ 
        H2 & top & (somente se necessário) \\ 
        F1 & top & Fusível 2410 500 mA \\ 
        C1 & top & Cap eletrolítico 100uF PTH \\ 
        J1 & top & Jumper ON / OFF da alimentação do GLV \\ 
        R5 & top & R0805 10k  \\ 
        LED3 & top & 0805 RED \\ 
        \hline
    \end{tabular}
\end{table}

\begin{table}[h]
    \centering
    \footnotesize 
    \caption{Validação.}
    \begin{tabular}{c|c}
        \textbf{Validação} & \textbf{Descrição} \\ 
        \hline
        Entrada & Fonte de bancada configurada em 15 V / 250 mA máximo aplicada em H1 através do chicote \\ 
        Saída & 15 V em J1, LED3 aceso ao selecionar jumper \\ 
        Medição & Entre J1 e o negativo de H1 \\
        \hline
    \end{tabular}
\end{table}

\subsection{Regulador 5 V}

\begin{table}[h]
    \centering
    \small 
    \caption{Componentes a soldar.}
    \begin{tabular}{c|c|l}
        \textbf{ID} & \textbf{Layer} & \textbf{Descrição} \\ 
        \hline
        C2, C5 & bottom & C0805 1uF \\ 
        C3, C4 & bottom & C0805 100nF \\ 
        U2 & top & Regulador MP1584EN PTH \\ 
        R4  & top & R0805 10k  \\ 
        LED2  & top & 0805 RED \\ 
        J4 & top & jumper selector do 5 V \\ 
        \hline
    \end{tabular}
\end{table}

\begin{table}[h]
    \centering
    \footnotesize 
    \caption{Validação.}
    \begin{tabular}{c|c}
        \textbf{Validação} & \textbf{Descrição} \\ 
        \hline
        Entrada & Fonte de bancada configurada em 15 V / 250 mA máximo aplicada em H1 através do chicote \\ 
        Saída & 5 V em J4, LED2 aceso ao selecionar jumper \\ 
        Medição & Entre J4 (inferior) e o negativo de H1 \\
        \hline
    \end{tabular}
\end{table}

\subsection{Regulador USB-C}

\begin{table}[h]
    \centering
    \small 
    \caption{Componentes a soldar.}
    \begin{tabular}{c|c|l}
        \textbf{ID} & \textbf{Layer} & \textbf{Descrição} \\ 
        \hline
        R10, R11 & bottom & R0805 5.1k \\ 
        H5  & top & Conector USB-C \\ 
        \hline
    \end{tabular}
\end{table}

\begin{table}[h]
    \centering
    \footnotesize 
    \caption{Validação.}
    \begin{tabular}{c|c}
        \textbf{Validação} & \textbf{Descrição} \\ 
        \hline
        Entrada & Conectar o USB-C no notebook \\ 
        Saída & 5 V em J4, LED2 aceso ao selecionar jumper \\ 
        Medição & Entre J4 (superior) e o negativo de H1 \\
        \hline
    \end{tabular}
\end{table}

\subsection{Regulador 3.3 V}

\begin{table}[h]
    \centering
    \small 
    \caption{Componentes a soldar.}
    \begin{tabular}{c|c|l}
        \textbf{ID} & \textbf{Layer} & \textbf{Descrição} \\ 
        \hline
        C6 & bottom & C0805 10uF \\ 
        C7 & bottom & C0805 22uF \\ 
        U3 & bottom & Regulador 3V3 \\ 
        J3 & bottom & jumper ON / OFF da fonte 3V3 \\ 
        R1 & top & R0805 4.7k \\ 
        LED1  & top & 0805 RED \\ 
        \hline
    \end{tabular}
\end{table}

\begin{table}[h]
    \centering
    \footnotesize 
    \caption{Validação.}
    \begin{tabular}{c|c}
        \textbf{Validação} & \textbf{Descrição} \\ 
        \hline
        Entrada & Conectar o USB-C no notebook \\ 
        Saída & 3.3 V em J3, LED1 aceso ao selecionar jumper \\ 
        Medição & Entre J3 e o negativo de H1 \\
        \hline
    \end{tabular}
\end{table}

\newpage

\section{Microcontrolador (MCU)}

Chicotes necessários: Nanofit 6 vias conectado a um ST-Link V2.
Verificar pinout antes de soldar o crimpar / soldar o conector.

\subsection{Circuitos Essenciais}

\begin{table}[h]
    \centering
    \small 
    \caption{Componentes a soldar.}
    \begin{tabular}{c|c|l}
        \textbf{ID} & \textbf{Layer} & \textbf{Descrição} \\ 
        \hline
        U5 & top & STM32U5, QFP48 package \\ 
        C17, C16, C24, C20, C18 & top & C0805 100nF (acoplamento) \\ 
        C14, C15 & bottom & C0805 1uF (acoplamento) \\ 
        C28 & top & C0805 4.7uF \\ 
        XP1 & top & Cristal 8 MHz PTH  \\ 
        C13, C19 & top & C0805 22pF  \\ 
        R13 & bottom & R0805 10k  \\ 
        C22 & bottom & C0805 100nF  \\ 
        B1 & top & RESET Button  \\ 
        R12 & bottom & R0805 10k  \\ 
        C21 & bottom & C0805 100nF  \\ 
        J5 & top & BOOT Jumper  \\ 
        \hline
    \end{tabular}
\end{table}

\begin{table}[h]
    \centering
    \footnotesize 
    \caption{Validação.}
    \begin{tabular}{c|c}
        \textbf{Validação} & \textbf{Descrição} \\ 
        \hline
        Entrada & Fonte de bancada configurada em 15 V / 250 mA máximo aplicada em H1 através do chicote \\ 
        Saída & Fonte não acusa curto, dreno de cerca de 20 mA do MCU estável \\ 
        \hline
    \end{tabular}
\end{table}

\subsection{Programação e Debug}

\begin{table}[h]
    \centering
    \small 
    \caption{Componentes a soldar.}
    \begin{tabular}{c|c|l}
        \textbf{ID} & \textbf{Layer} & \textbf{Descrição} \\ 
        \hline
        H7 & top & Nanofit 6 vias 180 graus \\
        J6 & top & Jumper seletor da alimentação do ST-Link \\
        B2 & bottom & DEBUG button \\
        R15 & bottom & R0805 10k \\
        C29 & bottom & C0805 100nF \\
        R17 & top & R0805 4.7k \\
        LED7 & top & 0805 RED \\
        \hline
    \end{tabular}
\end{table}

\begin{table}[h]
    \centering
    \footnotesize 
    \caption{Validação.}
    \begin{tabular}{c|c}
        \textbf{Validação} & \textbf{Descrição} \\ 
        \hline
        Entrada & Conectar USB-C ao notebook \\ 
        Firmware & Piscar LED7 quando B2 é acionado \\ 
        Saída & LED7 pisca ao acionar B2 \\ 
        \hline
        Entrada & Conectar USB-C ao notebook \\ 
        Firmware & Piscar LED7 quando um timer dispara, com breakpoint em debug na ISR \\ 
        Saída & LED7 pisca ao disparar timer, com breakpoint na ISR \\ 
        \hline
        Entrada & Conectar USB-C ao notebook \\ 
        Firmware & Escrever Hello World na UART conectada ao PC, exibir no terminal Putty \\ 
        Saída & Hello World aparece no terminal Putty do PC \\ 
        \hline
    \end{tabular}
\end{table}

\newpage

\section{Bloco da CAN} 
        
    %     \chapter{HISTÓRICO DE VERSÕES}
    %         \begin{center}
    %             \begin{longtable}{lllp{8cm}}
    %                 \arrayrulecolor{Tesla} 
    %                 \textbf{Data} & \textbf{Versão} & \textbf{Autor} & \textbf{Descrição} \\ \hline
    %                 \endfirsthead
                    
    %                 11/12/2025 & Primeira versão & João Megali & Primeira versão (Docs) \\ 
    %                 14/12/2025 & Primeira versão & Raphael Leivas  & Primeira versão (LaTeX) \\ 
    %             \end{longtable}
    %             \addtocounter{table}{-1}
    %         \end{center}
	\chapter*{}
            \begin{figure}
                \centering
                \vfill
                \includegraphics[width=0.7\linewidth]{logos/senóide verde 2024.png}
                \vfill
            \end{figure}
\end{document}
