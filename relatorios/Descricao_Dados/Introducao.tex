\chapter{Descrição dos Dados}

\section{Variáveis}

Cada arquivo CSV é composto de 59 colunas representando as 59 variáveis coletadas durante o teste. 
Cada uma delas está descrita na Tabela \ref{tab:variaveis}:

\begin{table}[h]
    \centering
    \footnotesize 
    \caption{Variáveis presentes nos CSVs.}
    \label{tab:variaveis}
    \begin{tabular}{l|c|l}
        \textbf{Variável} & \textbf{Unidade} & \textbf{Descrição} \\ 
        \hline
        timestamp & ms &  UNIX timestamp indicando o número de milissegundos de 1970 \\ 
        t & ms & milissegundos passados desde a primeira medição recebida no dia \\ 
        THROTTLE & 0 - 1000 & faixa de 0 a 1000 indicando o nível de aceleração do piloto \\ 
        BRAKE & 0 - 1 & faixa de 0 a 1000 indicando o nível de aceleração do piloto \\ 
        ECU\_MODE\_ID & 0 - 4 & valor 1 indica modo Enduro, valor 2 indica Arrancada \\ 
        TORQUE\_GAIN & \% & ganho de torque configurado na ECU \\ 
        TORQUE\_REF\_LEFT\_MOTOR & 0 - 1000 & Faixa indicando a referência de torque no motor esquerdo \\ 
        TORQUE\_REF\_RIGHT\_MOTOR & 0 - 1000 & Faixa indicando a referência de torque no motor direito \\
        LEFT\_MOTOR\_RPM & rpm & RPM medido no motor esquerdo pelo inversor \\
        LEFT\_MOTOR\_TORQUE & Nm & Torque em Nm medido no motor esquerdo pelo inversor \\
        LEFT\_MOTOR\_CURRENT & A & Corrente em A medida no motor esquerdo pelo inversor \\
        RIGHT\_MOTOR\_RPM & rpm & RPM medido no motor direito pelo inversor \\
        RIGHT\_MOTOR\_TORQUE & Nm & Torque em Nm medido no motor direito pelo inversor \\
        RIGHT\_MOTOR\_CURRENT & A & Corrente em A medida no motor direito pelo inversor \\
        REAR\_LEFT\_WHEEL\_SPEED & rpm & Velocidade em RPM pela roda fônica traseira esquerda \\
        REAR\_RIGHT\_WHEEL\_SPEED & rpm & Velocidade em RPM pela roda fônica traseira direita  \\
        FRONT\_LEFT\_WHEEL\_SPEED & km/h & Velocidade em km/h medida pela roda fônica dianteira esquerda \\
        FRONT\_RIGHT\_WHEEL\_SPEED & km/h & Velocidade em km/h medida pela roda fônica dianteira direita \\
        ACCEL\_LONGITUDINAL & G & Aceleração longitudinal do carro em G \\
        ACCEL\_LATERAL & G & Aceleração lateral do carro em G \\
        ACCEL\_NORMAL & G & Aceleração normal (vertical) do carro em G \\
        VEL\_ANGULAR\_YAW & ° / s & velocidade angular do ângulo de guinada (yaw) \\
        VEL\_ANGULAR\_ROLL & ° / s & velocidade angular do ângulo de rolagem (roll, bank angle) \\
        VEL\_ANGULAR\_PITCH & ° / s & velocidade angular do ângulo de arfagem (pitch) \\
        ELETROBUILD\_TEMPERATURE & °C & Temperatura em Celsius do eletrobuild medida pelo IMU \\
        STACK\_i\_CELL\_j & V & Tensão em Volts da célula $j$ na stack $i$ \\
        MAX\_VOLTAGE & V & Tensão da célula com maior tensão em Volts medido pelo BMS \\
        MIN\_VOLTAGE & V & Tensão da célula com menor tensão em Volts medido pelo BMS \\
        TOTAL\_VOLTAGE & V & Tensão total do acumulador (soma de todas as células) \\ 
        SHUNT\_CURRENT & A & corrente medida no shunt, sem subtrair offset \\
        BMS\_MODE\_ID &  & ID do modo do BMS \\ 
        BMS\_ERROR\_ID &  & ID do erro do BMS \\
        AIR\_P & 0 - 1 & status do AIR positivo (0 aberto, 1 fechado) \\
        AIR\_N & 0 - 1 & status do AIR negativo (0 aberto, 1 fechado) \\
        VEHICLE\_SPEED & km/h & Velocidade do carro, tomada a partir da média dos RPMs dos motores \\
        HODOMETRO & m & Distância em metros percorrida desde o início do teste \\
        \hline
    \end{tabular}
\end{table}

\section{Arquivos}

A Tabela \ref{tab:arquivos} mostra o conteúdo de cada arquivo, para facilitar 
a identificação.

\begin{table}[h]
\centering
\footnotesize 
\caption{Conteúdo de cada arquivo CSV.}
\label{tab:arquivos}
\begin{tabularx}{\textwidth}{l|c|c|l}
\textbf{Arquivo} & \textbf{Intervalo de tempo} & \textbf{Distância (m)} & \textbf{Contexto Adicional} \\
\hline
miguelito1\_enduro1.csv & 11:31:30 - 11:33:50 & 409  & Sem erros \\
miguelito1\_enduro2.csv & 11:43:00 - 11:43:45 & 24   & Arrancou e parou logo em seguida: falha de subtensão \\
miguelito1\_enduro3.csv & 11:45:15 - 11:47:45 & 322  & Falha de subtensão no final \\
miguelito1\_enduro4.csv & 11:58:15 - 12:01:00 & 427  & Falha de subtensão no final \\
miguelito1\_enduro5.csv & 12:02:50 - 12:04:30 & 328  & Carro rodou no final \\
miguelito1\_enduro6.csv & 12:05:50 - 12:11:30 & 1465 & Sem erros \\
miguelito1\_enduro7.csv & 12:15:20 - 12:16:30 & 230  & Falha no pinhão da corrente no final \\
mike1\_enduro1.csv      & 13:25:20 - 13:32:00 & 1622 & Sem erros \\
mike1\_enduro2.csv      & 13:40:00 - 13:42:00 & 459  & Falha de subtensão no final \\
mike1\_enduro3.csv      & 13:42:30 - 13:43:00 & 65   & Arrancou e teve falha de subtensão \\
mike1\_enduro4.csv      & 13:44:10 - 13:46:15 & 388  & Falha de subtensão no final \\
miguelito2\_enduro1.csv & 17:03:30 - 17:06:15 & 425  & Falha no final (sem leituras do BMS) \\
miguelito2\_enduro2.csv & 17:10:45 - 17:11:45 & 217  & Falha no final (sem leituras do BMS) \\
\hline
\end{tabularx}
\end{table}

\section{Considerações Finais}

A ECU está escrevendo velocidade em km/h igual a zero tanto da leitura do inversor quanto das rodas fônicas quando o piloto tira o pé do acelerador. 
Por consequência, a velocidade oscila bastante entre a velocidade real e zero quando o piloto para de acelerar
medições do IMU continuam corretas, então ainda é possível ter uma noção da velocidade sem o piloto acelerar
vamos olhar uma solução para isso ainda.

Além disso, esse efeito fez com que a distância total calculada pela integração numérica do hodômetro seja menor que a real.
No entanto, estimamos que não seja uma diferença muito grande.


Processamos os dados para sincronizar os pacotes entre si, e removemos os intervalos de tempo em que o carro estava parado / em falha de subtensão (mas os instantes que levaram às falhas estão todos ainda). Se precisarem de mais intervalos de tempo podem falar que olhamos aqui na base completa

Se tiverem interesse, dados puros, binários e segregados por pacote podem ser encontrados no repositório, junto com os scripts em Python que processaram eles para gerar o arquivo zip https://github.com/Tesla-UFMG/Data-Analysis

Se notarem algum resultado inesperado em uma prova, teste as outras provas também para ver se o padrão se repete. se acontecer em todos, pode ser consequência da medição e fale com a gente

\vspace{2cm}

\noindent go tesla 
