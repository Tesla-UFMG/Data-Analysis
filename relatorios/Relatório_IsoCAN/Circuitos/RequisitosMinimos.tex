\chapter{Requisitos mínimos} 

    \section{Funções}
        Primeiro foi definido qual as funções necessárias para o bom funcionamento da IsoCAN.
        \begin{itemize}
            \item Isolamento galvânico para o circuito de CAN dos inversores, então o barramento deve entrar de um lado da placa;
            \item Ter 6mm entre o lado considerado de high voltage e o de low voltage;
            \item Conversor dc-dc isolado para alimentar o circuito de alta tensão;
            \item Fazer a informação chegar na ecu com o mínimo de atraso possível e garantir que as informações não trazem e sejam coerentes.
        \end{itemize}

        Após ter definidos as funções da IsoCAN é possível selecionar quais sensores serão necessários, e quais interfaces estarão presentes.
    
    \section{Entradas e saídas da placa} 
        Por fim, foi registrado as portas da placa, para já permitir o início do chicote.
        \begin{itemize}
            \item Alimentação negativa e positiva;
            \item 2 portas (CAN High e CAN Low) para barramento CAN de entrada do inversor. Total: 2;
            \item 2 portas (CAN High e CAN Low) para barramento CAN da saída para a ECU. Total: 2;
            \item Total: 6 portas.
        \end{itemize}

    \section{Conectores}
         Com o projeto de padronização de conectores, a intenção de garantir que nenhuma confecção possa ser feita de maneira incorreta e a necessidade de um conector de fácil conexão, crimpagem e seguro no quesito chicote, fez se as seguintes escolhas:
         \begin{itemize}
         \item Nanofit de 4 vias: gnd, alimentação, can high e can low
         \item Nanofit de 2 vias: can high e can low
         \end{itemize}
