\chapter{Hardware}
\section{Circuitos gerais}
    \subsection{Alimentação}
A IsoCAN deve ser alimentada com uma tensão de alimentação entre 12 a 16V, a partir de um conector nanofit de quatro vias. A pinagem do nanofit de quatro vias segue o padrão igual à outras placas de Controle, com o pino 1 sendo GND e o pino 2 a tensão positiva da bateria. O  circuito de proteção também é padrão, ele pode ser visto na \autoref{fig:Alimentação}. Um diodo protege contra ligação invertida enquanto o capacitor eletrolítico filtra ruídos da alimentação e o fusível de 1A protege contra curtos-circuitos. 
        \begin{figure}[H]
            \centering
            \includegraphics[width=0.95\linewidth]{Circuitos/imagens/Alimentação .png}
            \caption{\footnotesize Circuito de alimentação}
            \label{fig:}
        \end{figure}

O regulador de tensão LM1117-5.0 abaixa e estabiliza sua tensão entrada para 5V, para alimentar os circuitos do protocolo de comunicação CAN. Esse regulador também é responsável pela alimentação do LED

\section{Circuitos de protocolos de comunicação}
O protocolo CAN é feito para realizar a comunicação entre unidades de controle eletrônico, sendo serial, diferencial e dando prioridade de envio de mensagens a nós com endereços mais baixos.

Para possibilitar a recepção e envio das mensagens de CAN no MCU, mesmo que este tenha suporte ao protocolo, é necessária a utilização de um transceiver. Esse equipamento realiza a conversão entre o sinal diferencial High-Low e os pinos RX e TX, que podem ser eletricamente ligados ao MCU de forma direta. A ideia é manter o transceiver atual da ECU sem o isolamento e, entre o terminal dos inversores, inserir esta placa que apresentará dois transceivers (um para cada seção do barramento). O transceiver cujo terminal se conecta aos inversores será alimentado por um conversor DC-DC isolado \autoref{fig:5V Isolado} com distância mínima entre contatos a serem isolados de 6 mm, requerida pelo atual regulamento.

    \subsection{CAN inversores}
A ECU é um dos terminais da CAN de controle, fazendo-se necessária a instituição de dois resistores de aproximadamente 60 Ω para evitar reflexões do sinal. Também foi implementado um filtro para o sinal, utilizando capacitores de 50 nF. O choke é um componente indutivo que filtra componentes de modo comum em alta frequência nos contatos High e Low.

        \begin{figure}[H]
            \centering
            \includegraphics[width=0.95\linewidth]{Circuitos/imagens/5V isolado.png}
            \caption{\footnotesize Circuito de alimentação isolado}
            \label{fig:5V Isolado}
        \end{figure} 
        
        \begin{figure}[H]
            \centering
            \includegraphics[width=0.95\linewidth]{Circuitos/imagens/CANInversor.png}
            \caption{\footnotesize Circuito Can - Lado Inversores}
            \label{fig:Can ECU}
        \end{figure} 

    \subsection{Lógica arbitrária}
Entre os pinos TX e RX dos transceivers, há um circuito lógico cujo objetivo é emular a lógica wired AND do barramento CAN e impedindo o travamento do dispositivo. O circuito de referência para implementação de um repetidor de CAN que foi utilizado como inspiração se encontra em \autoref{fig:Lógica arbitrária}   
        \begin{figure}[H]
            \centering
            \includegraphics[width=0.95\linewidth]{Circuitos/imagens/Lógica arbitrária.jpeg}
            \caption{\footnotesize Lógica arbitrária}
            \label{fig:Lógica arbitrária}
        \end{figure} 
        
    \subsection{CAN ECU}    
Esse circuito é basicamente um espelho do circuito do lado dos inversores. Vale ressaltar que eles ainda estão sendo testados
        \begin{figure}[H]
            \centering
            \includegraphics[width=0.95\linewidth]{Circuitos/imagens/CanECU.png}
            \caption{\footnotesize Circuito Can - Lado ECU}
            \label{fig:Can ECU}
        \end{figure}
    
\chapter{Layout}
        Definidos todos os circuitos, poderá ser realizado o desenho da PCB. O layout está sendo realizado em duas layers e a placa possuirá 4 furos para fixação. Esses furos são vias com diâmetro de 7mm e diâmetro do furo de 5mm, isso foi realizado com a intenção de evitar que algum dano na proteção da placa criado por um parafuso gere algum curto com qualquer circuito da placa. 