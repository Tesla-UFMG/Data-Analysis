\chapter{INTRODUÇÃO}
A Unidade de Controle Eletrônico (ECU) é a placa responsável por controlar o torque dos dois motores elétricos a partir de inversosres de frequência. A ECU se comunica com os inversores através de um barramento CAN exclusivo.\par

Nas antigas versões da ECU o circuito do terminal CAN manteve-se conectado diretamente aos inversores, dessa forma, era parte do \textit{Tractive System} (TS). Como, por regulamento, todos os componentes do carro que têm contato com o TS devem ficar atrás do \textit{firewall} do carro, trouxe dificuldades para o manejamento de espaço físico em decorrência da grande quantidade de chicote conectado a ECU e da pouca visibilidade e difícil acesso ao compartimento eletrônico, onde a ECU está atualmente localizada. \par

Para solucionar esses problemas, foi idealizado o projeto IsoCAN, que tem como função principal isolar o barramento do protocolo de comunicação CAN de alta tensão entre a ECU e os inversores. Dessa forma, deve otimizar o espaço no compartimento eletrônico do carro, diminuir a interferência eletromagnética, problema que enfrentamos muito ao longo dos anos, e contribuir para a diminuição do chicote. Essa mudança trará mais segurança e integridade para os componentes eletrônicos, diminuição dos travamentos da ECU e, além disso, mais conforto para os membros no processo de Debug da placa. \par

A idealização do projeto da IsoCAN é de longa data, já que anteriormente era um projeto de Chicote, mas no começo na temporada de 2024/2025 passou a ser de controle. A placa criada se baseou numa placa desenvolvida pela \textit{Texas Instruments}, chamada \textit{TIDA-01487}, que é um \textit{Isolated CAN Flexible Data Rate Repeater Refence Design} (em tradução livre: Projeto de Referência de Repetidor de Taxa de Dados Flexível Isolado). Este projeto de referência de repetidor CAN FD isolado adiciona isolamento elétrico entre dois segmentos de barramento CAN. Os CAN frames em cada lado do segmento de barramento são repetidos para o outro lado.