\chapter{Metodologia}
\section{Pesquisa e levantamento de requisitos do projeto}
Inicialmente, foi desenvolvido um estudo sobre o projeto realizado na temporada de 2025, junto a uma pesquisa sobre o protocolo de comunicação CAN e conversas com ex-membro. Nessa fase, todas as fontes de pesquisa foram documentadas em um relatório bibliográfico.  

A partir disso, têm-se um entendimento dos componentes necessários para a montagem da placa e os desafios que serão enfrentados durante o seu desenvolvimento. Além disso, uma projeção financeira foi realizada a partir dos componentes apresentados no projeto anterior. 

\section{Dimensionamento dos circuitos e simulação no LTspice}

Em seguida, os circuitos do projeto serão desenvolvidos e simulados no software \textit{LTspice}, designado para simular circuitos eletrônicos. Nessa etapa, calcula-se uma alteração nos componentes e circuitos da IsoCAN 2025, e também, um entendimento da causa de não conclusão da placa. Essas mudanças serão documentadas em um relatório técnico.

\section{Design da PCB}

Logo após o dimensionamento, o esquemático dos circuitos será feito através da plataforma \textit{EasyEDA}, no qual também será realizado o design da placa de circuito impresso (PCB) com 2 camadas (layers). Nessa etapa, também será desenvolvido um relatório parcial da IsoCAN.

\section{Solda e Validação}

Com o layout da PCB pronta e impressa, ocorrerá a montagem, e então, a solda da placa. Assim, se inicia a fase de testes.

Na primeira etapa da validação serão feitos testes em bancada da IsoCAN separadamente, com a ECU e a Unidade de Medição Inercial (IMU), e também, com a ECU e os inversores. Também existe a possibilidade de valida-la com um Arduino, com um código desenvolvido em temporadas anteriores e disponibilizado no GitHub. Dessa forma, cada especificação da placa será avaliada de forma individual. Após esse período, deve ser escrito o relatório final do projeto. 

Por fim, na última etapa do projeto e da validação, a IsoCAN será testada no carro com a ECU. Assim, será verificada a interação da placa com os outros sistemas do carro e o seu real desempenho em condições reais.
