\chapter{Bibliografia}
Logo abaixo estão indicados todos os arquivos utilizados na etapa de pesquisa do projeto e uma explicitação dos seus conteúdos.

\section{Protocolo CAN}
    \begin{itemize}
    \item SOUZA, Paulo Vítor de. \textbf{\textit{Estudo e Elaboração de uma Rede CAN para Aplicação em um Sistema Automotivo}}. Divinópolis: CEFET-MG, 2019. Disponível em: \url{https://www.eng-mecatronica.divinopolis.cefetmg.br/wp-content/uploads/sites/195/2019/12/Paulo-V%C3%ADtor-de-Souza1.pdf}. Acesso em: 6 out. 2025.
    \\
    \\
    Aborda a concepção e implementação de uma rede CAN para uso em sistemas automotivos. Apresenta os fundamentos do protocolo CAN, as tecnologias de física de camada e as técnicas de medição de tempo, e descreve a implementação prática da rede em ambiente veicular, incluindo testes de comunicação, análise de erros e ajustes necessários para garantir confiabilidade e integridade dos dados.
    \\
    \item NASCIMENTO, L. de C. \textbf{\textit{Protocolo de comunicação CAN e suas aplicações na indústria automobilística}}. 
    Itatiba, 2006. 51 f. Trabalho de Conclusão de Curso (Engenharia Elétrica) — Universidade São Francisco, 2006. 
    Disponível em: \url{https://lyceumonline.usf.edu.br/salavirtual/documentos/1598.pdf}. 
    Acesso em: 6 out. 2025.
    \\
    \\
    Explica o funcionamento do protocolo CAN, sua estrutura física e lógica, e as vantagens em aplicações automotivas. Aborda a troca de dados entre módulos eletrônicos de um veículo e destaca os benefícios de confiabilidade, velocidade e redução de cabos.
    \\
    \item BARBOSA, Luiz Roberto Guimarães. 
    \textit{\textbf{Rede CAN}}. Belo Horizonte: Escola de Engenharia da Universidade Federal de Minas Gerais, 2003. 
    Disponível em: \url{https://pdfcoffee.com/rede-can-10-pdf-free.html}. 
    Acesso em: 6 out. 2025.
    \\
    \\
    Apresenta os fundamentos teóricos da rede CAN, incluindo histórico, camadas do protocolo, estrutura dos quadros de dados, arbitragem, verificação de erros e níveis físico e de enlace. Explica o funcionamento elétrico e lógico do barramento e suas aplicações em veículos e sistemas industriais.
    \\
    \end{itemize}

\section{Projeto de referência TIDA-01487 (TI)}

\begin{itemize}

    \item MAUER, Thomas. \textbf{\textit{Isolated CAN FD Repeater Reference Design (Rev. A)}}. Dallas: Texas Instruments Incorporated, 2017. Revisado em abr. 2018. Disponível em: \url{https://www.ti.com/tool/TIDA-01487}. Acesso em: 6 out. 2025.
    \\
    \\
    Apresenta o projeto de um repetidor CAN FD isolado, com foco em sistemas automotivos e industriais. Descreve o princípio de funcionamento, o esquema do circuito, considerações de layout de PCB e resultados de testes de desempenho. O projeto utiliza componentes da Texas Instruments para demonstrar isolamento galvânico e alta imunidade a ruídos eletromagnéticos.
    \\
    \item TEXAS INSTRUMENTS INCORPORATED. \textbf{\textit{TIDA-01487 BOM (Rev. A)}}. Dallas: Texas Instruments, 2018. Disponível em: \url{https://www.ti.com/tool/TIDA-01487}. Acesso em: 6 out. 2025.
    \\
    \\
    Apresenta a Bill of Materials (BOM) do projeto TIDA-01487, listando todos os componentes utilizados (resistores, capacitores, diodos, circuitos integrados e conectores) com suas especificações, fabricantes e códigos de referência.
    \\
    \item MAUER, Thomas. \textbf{\textit{TIDA-01487 Schematic and Block Diagram (Rev. A)}}. Dallas: Texas Instruments, 2017. Revisado em mar. 2018. Disponível em: \url{https://www.ti.com/tool/TIDA-01487}. Acesso em: 6 out. 2025.
    \\
    \\
    Contém os diagramas elétricos e de blocos do projeto TIDA-01487. O arquivo detalha todas as conexões entre os módulos do repetidor CAN FD, os componentes principais e o fluxo de sinal.
    \\
\end{itemize}

\section{ECU e Projeto IsoCAN 2025}

\begin{itemize}
    \item TELLES, Felipe. \textbf{\textit{Relatório ECU 3.1}}. Belo Horizonte: Tesla UFMG, 2022. Disponível em: \url{https://drive.google.com/file/d/1t6ggPlPl4_9SJGHyuEPIxJFTBWkUa3P5/view}. Acesso em: 6 out. 2025.
    \\
    \\
    Relatório técnico que documenta o desenvolvimento da ECU 3.1. Descreve a arquitetura de hardware, dimensionamento dos circuitos, protocolos de comunicação, além de testes e problemas encontrados.
    \\
    \item TELLES, Felipe; FERREIRA, Hyan Carvalhido; LEANDRO, Letícia Ferreira. 
    \textbf{\textit{Relatório ECU Iso-CAN}}. Jan. 2025. Relatório técnico interno. 
    Disponível em: \url{https://docs.google.com/document/d/1aM6-yjwkl9k6RZ-6qEqv3EnGpGOY4zd0nhwuVVRVSbs/edit?tab=t.0}. 
    Acesso em: 6 out. 2025.
    \\
    \\
    Documenta o desenvolvimento e os testes da ECU IsoCAN. Descreve a arquitetura de hardware, os circuitos utilizados, as estratégias de isolamento, os cronogramas de execução, resultados parciais de testes e as recomendações para melhorias futuras da placa.
    \\
    \item FERREIRA, Hyan Carvalhido; LEANDRO, Letícia Ferreira. 
    \textit{\textbf{Relatório bibliografia e metodologia ECU ISO-CAN}}. 
    31 out. 2024. Relatório técnico interno. 
    Disponível em: \url{https://docs.google.com/document/d/1Ew8D81tPBSXz3-KE_76VmYSTDrITNRpK70jUHvVGcbw/edit?tab=t.0}. 
    Acesso em: 6 out. 2025.
    \\
    \\
    Documenta a revisão bibliográfica e a metodologia usada no desenvolvimento da ECU IsoCAN 2025. Descreve melhorias em relação às versões anteriores (ECU 3.1 e 3.2), ajustes de componentes, testes e validações. Inclui também as principais referências técnicas e datasheets usados no projeto.
    \\
    \item \textit{BOM ISO-CAN Review}. Belo Horizonte: Tesla UFMG, 2025. Revisado em 10 set. 2025. Lista de materiais (Bill of Materials) do projeto ECU Iso-CAN.
    \\
    \\
     Apresenta a Bill of Materials (BOM) do projeto IsoCAN 2025, listando todos os componentes utilizados (resistores, capacitores, circuitos integrados e conectores) com suas especificações, fabricantes e códigos de referência.
    \\  
\end{itemize}