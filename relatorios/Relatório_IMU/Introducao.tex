\chapter{INTRODUÇÃO}

Para a implementação dos controles dinâmicos, é fundamental que a equipe assegure que os sensores estejam devidamente calibrados e que suas medições sejam coerentes. Com esse objetivo, a equipe de Controle, em conjunto com a equipe de Aquisição de Dados, realizou testes utilizando o sensor desenvolvido internamente pela equipe, no caso, o IMU reserva. 

O IMU é responsável por obter a taxa de guinada real do veículo, isto é, a velocidade angular com que o veículo rotaciona em torno de seu eixo vertical. Essa informação é essencial para a vetorização de torque, uma vez que ela é utilizada para calcular o erro que será passado para o controlador.

A Figura \ref{fig:foto-imu} mostra uma imagem da PCB do IMU usada no teste.

\begin{figure}[h]
    \centering
    \includegraphics[width=0.8\textwidth]{images/foto-imu.png} 
    \caption{IMU.}
    \label{fig:foto-imu}
\end{figure}