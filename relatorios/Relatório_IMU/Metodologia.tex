\chapter{Metodologia}

Para verificar a precisão das medições do sensor, foi necessário determinar previamente um valor teórico de referência, que pudesse ser comparado ao valor efetivamente medido.
Para derivar a expressão do valor teórico foi preciso recorrer ao Movimento Circular Uniforme. Na Figura \ref{fig:mcu} conseguimos ver a trajetória no formato de um círculo em vermelho com raio $R$, com uma variação angular e uma variação linear $s$.  

\begin{figure}[h]
    \centering
    \includegraphics[width=0.5\textwidth]{images/mcu.png} 
    \caption{Movimento Circular Uniforme.}
    \label{fig:mcu}
\end{figure}


A velocidade angular é definida como a variação angular dividida pelo tempo. Em uma volta
completa, o corpo percorre $2\pi$ radianos em um período $T$.

\begin{equation}
\omega = \frac{\Delta\theta}{\Delta t} = \frac{2\pi}{T} \label{eq:vel_angular}
\end{equation}

Já a velocidade linear é dada pela distância linear percorrida dividida pelo tempo. Assim, em
um movimento circular, a distância percorrida é o comprimento da circunferência, isto é, $2\pi R$,
também dividido pelo período $T$.

\begin{equation}
v = \frac{\Delta s}{\Delta t} = \frac{2\pi R}{T} \label{eq:vel_linear}
\end{equation}

Ao analisar essas duas definições em conjunto, obtém-se a relação clássica entre velocidade
linear e velocidade angular que é dada pela Equação \ref{eq:relacao_v_omega}.

\begin{equation}
v = \omega R \label{eq:relacao_v_omega}
\end{equation}

Portanto, a relação que será utilizada nos testes para calcular o valor teórico será derivada da
Equação \ref{eq:relacao_v_omega}, tendo em vista a definição da taxa de guinada que foi apresentada na introdução.

\begin{equation}
\dot{\Psi} = \frac{v}{R} \label{eq:taxa_guinada}
\end{equation}

A distância total do percurso é obtida multiplicando-se o comprimento da circunferência pelo
número de voltas realizadas.

\begin{equation}
D = 2\pi R \cdot \un{(número de voltas)} \label{eq:distancia_total}
\end{equation}

A velocidade será obtida pelo o cálculo da velocidade média desenvolvida no percurso. 

\begin{equation}
v = \frac{D}{\Delta t} \label{eq:vel_media}
\end{equation}

O erro em porcentagem entre o valor teórico, calculado pela Equação 4, e pelo obtido com o
sensor do IMU será dado pela Equação \ref{eq:erro_percentual}.

\begin{equation}
\un{erro} = \frac{\un{medido} - \un{teórico}}{\un{teórico}} \cdot 100 \label{eq:erro_percentual}
\end{equation}

Para a realização dos testes, o IMU foi fixado no veículo de um dos membros da equipe, de
modo a mantê-lo o mais imóvel possível e minimizar o efeito de vibrações que poderiam
comprometer as medições. Até o momento, não foi possível testar o sensor no protótipo da equipe
Fórmula Tesla UFMG.

Como orientação ao piloto, foi informado que não seria necessária uma alta velocidade,
entretanto, seria importante manter uma velocidade constante e evitar o uso do freio. Os testes foram
realizados nas rotatórias da própria UFMG, e os resultados serão apresentados a seguir.