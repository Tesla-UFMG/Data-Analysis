\chapter{Teste}

A configuração do IMU encontra-se na Tabela \ref{tab:config-imu}.
Os fatores de escala devem ser multiplicados pelos dados crus do IMU salvos nos CSVs (somente o giroscópio nesse caso).

\begin{table}[h]
    \centering
    \caption{Configuração do IMU.}
    \label{tab:config-imu}
    \begin{tabular}{lc}
        \toprule
        \textbf{Parâmetro} & \textbf{Valor} \\
        \midrule
        Fator de escala do acelerômetro & 0,244 / 1000 \\
        Fator de escala do giroscópio & 70 / 1000 \\
        Eixo de análise & Z (giroscópio) \\
        Alimentação & GLV de Pilha (16V) \\
        \bottomrule
    \end{tabular}
\end{table}

O primeiro teste realizado foi na rotatória da INOVA, próximo do CPH. Nesse sentido, vale destacar que a via não era asfaltada o que gerou trepidações. Já o segundo teste foi realizado na rotatória da reitoria que é asfaltada. A Figura \ref{fig:rotatorias-teste} mostra o mapa das rotatórias de teste.

\begin{figure}[h]
    \centering
    \includegraphics[width=0.8\textwidth]{images/rotatorias-teste.png} 
    \caption{Rotatórias de teste: (a) primeiro teste próximo do INOVA (CPH) e (b) segundo teste próximo da reitoria .}
    \label{fig:rotatorias-teste}
\end{figure}

Os dados dos testes estão apresentados na Tabela \ref{tab:resultados_rotatorias}. A distância total foi calculada utilizando a Equação \ref{eq:distancia_total} e a velocidade média a Equação \ref{eq:vel_media}.

\begin{table}[h]
    \centering
    \caption{Resultados dos Testes nas Rotatórias.}
    \label{tab:resultados_rotatorias}
    \begin{tabular}{lcc}
        \toprule
        \textbf{} & \textbf{INOVA} & \textbf{REITORIA} \\
        \midrule
        Raio (m) & 15 & 7,25 \\
        Tempo (s) & 187 & 25 \\
        Número de voltas & 5 & 2 \\
        Distância total (m) & 565,4866 & 91,357 \\
        Velocidade média (m/s) & 3,0239 & 3,654 \\
        \bottomrule
    \end{tabular}
\end{table}

Os dados obtidos no teste podem ser acessados pelo Git da equipe Fórmula Tesla UFMG \url{https://github.com/Tesla-UFMG/Data-Analysis/tree/master/data/validacao_imu_volta}.

\chapter{Resultados}

A Tabela \ref{tab:calculo_taxa} mostra os valores de taxa de guinada e erro calculados nos testes.

\begin{table}[h]
    \centering
    \caption{Cálculo da taxa de guinada e do erro}
    \label{tab:calculo_taxa}
    \begin{tabular}{lcc}
        \toprule
        \textbf{} & \textbf{INOVA} & \textbf{REITORIA} \\
        \midrule
        Taxa de guinada teórico (graus/s) & 11,55 & 28,8 \\
        Taxa de guinada IMU (graus/s) & 14,2 & 30,88 \\
        erro (\%) & 22,9 & 7 \\
        \bottomrule
    \end{tabular}
\end{table}

Vale destacar que o erro observado no teste realizado no INOVA pode ser atribuído ao grande número de voltas, uma vez que a velocidade tende a variar mais ao longo do percurso, além do pavimento, que introduz ruído adicional na medição. Ao considerar apenas uma janela correspondente a uma única volta e refazer os cálculos, obtém-se os dados da Tabela \ref{tab:volta_unica}

\begin{table}[h]
    \centering
    \caption{Cálculo da taxa de guinada e erro considerando apenas uma volta}
    \label{tab:volta_unica}
    \begin{tabular}{lc}
        \toprule
        \textbf{Parâmetro} & \textbf{INOVA (1 Volta)} \\
        \midrule
        Raio (m) & 15 \\
        Tempo (s) & 26 \\
        Número de voltas & 1 \\
        Distância total (m) & 94,247 \\
        Velocidade média (m/s) & 3,624 \\
        Taxa de guinada teórico (graus/s) & 13,846 \\
        Taxa de guinada IMU (graus/s) & 14,1 \\
        erro (\%) & 1,83 \\
        \bottomrule
    \end{tabular}
\end{table}

\chapter{Conclusão}

O sensor da IMU mostrou-se bastante preciso e confiável. Vale destacar, porém, que a aplicação de um filtro sobre sua medição é recomendada para atenuar o ruído, uma vez que a componente integral do controlador da vetorização de torque tende a acumular esses erros ao longo do tempo.
