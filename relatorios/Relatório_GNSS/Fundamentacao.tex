\chapter{Fundamentação Teórica}

Sinais de GNSS são sinais de 1.5 GHz extraídos de antenas. Por serem de alta frequência, 
temos que estudar práticas de circuitos de RF (Radio Frequency) para garantir o funcionamento correto do circuito.
Além disso, sinais de GNSS são muito fracos, o que exige cuidados adicionais no projeto do circuito.

Toda a fundamentação teórica está baseada nos seguintes documentos principais:

\begin{itemize}
    \item Manual do Hardware do Teseo-LIV3f da ST: \url{https://drive.google.com/file/d/1f-_5UErfYuVgv-IKlNlQNE1Cbe42LFDT/view?usp=drive_link}
    \item White Paper de boas práticas para layout de circuitos de RF: \url{https://drive.google.com/file/d/17EAXY1Gev_J_ZbgDPay-ZY9WCtNQap2y/view?usp=drive_link}
    \item Guia da ST para circuitos RF: \url{https://drive.google.com/file/d/15d90DjYcKHO8I-ajw5f05wGQfgVGjobQ/view?usp=drive_link}
    \item Tese de análise de antenas de patch cerâmica: \url{https://drive.google.com/file/d/1AVxLt7DhbOgQ1vpqmCdnyk6QVYBsWJfR/view?usp=drive_link}
\end{itemize}

Um White Paper é um documento publicado por uma empresa que aborda os detalhes técnicos de como 
um determinado produto é desenvolvido, sendo assim uma excelente fonte de informações 
para a equipe.

\section{Antenas Patch Cerâmicas}

A antena patch cerâmica é um tipo de antena de microfita compacta, formada por um elemento metálico (patch) sobre um substrato cerâmico de alta permissividade dielétrica e um plano de terra na face oposta.
A estrutura básica está exibida na Figura \ref{fig:antena_ceramica}.

\begin{figure}[h]
    \centering
    \includegraphics[width=0.6\textwidth]{images/antena_ceramica.png} 
    \caption{Antena Patch Cerâmica.}
    \label{fig:antena_ceramica}
\end{figure}

A partir dos valores da estrutura física da antena, ela apresenta uma determinada resposta em frequência, 
de modo que ela consegue capturar e transmitir sinais em uma faixa de frequência específica.
No caso do GNSS, a antena deve ser projetada para operar na faixa de 1.5 GHz, que é a frequência
em que os sinais de satélites GNSS são transmitidos.

A Figura \ref{fig:resposta_antena} mostra a resposta em frequência de uma antena patch cerâmica típica para GNSS.
Note que ela opera apenas na faixa de interesse de 1.5 GHz em que os satélites 
enviam os sinais. Além disso, esse comportamento evita interferência de outros sinais 
em frequências diferentes, como o LoRa que opera em 900 MHz.

\begin{figure}[h]
    \centering
    \includegraphics[width=0.75\textwidth]{images/resposta_antena.png} 
    \caption{Resposta em frequência de uma antena patch cerâmica para GNSS.}
    \label{fig:resposta_antena}
\end{figure}

Antenas de patch cerâmicas podem ainda ser classificadas como ativas ou passivas:

\begin{itemize}
    \item Ativas: apresentam amplificador LNA embutido na antena, de modo que o sinal 
    enviado à placa já chega amplificado, e basta enviá-lo a um módulo GNSS (no nosso, caso, o 
    Teseo-LIV3f) através de um filtro pi (será explicado mais a frente).
    \item Passivas: não possuem amplificador embutido, de modo que o sinal deve ser 
    amplificado na própria placa antes de chegar ao módulo GNSS.
\end{itemize}

O amplificador LNA comumente usado é o MAX2659 \url{https://www.digikey.com.br/pt/products/detail/analog-devices-inc-maxim-integrated/MAX2659ELT-T/2062078}.
No seguinte link uma antena de patch cerâmica ativa é desmontada, exibindo o circuito interno 
que amplifica o sinal \url{https://www.youtube.com/watch?v=s-jFprdDcM4}. 

\section{Circuitos RF}

Uma série de cuidados devem ser tomados no projeto de cirucitos RF, listados abaixo:

\begin{itemize}
    \item 
\end{itemize}




