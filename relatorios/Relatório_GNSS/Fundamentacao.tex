\chapter{Fundamentação Teórica}

Sinais de GNSS são sinais de 1.5 GHz extraídos de antenas. Por serem de alta frequência, 
temos que estudar práticas de circuitos de RF (Radio Frequency) para garantir o funcionamento correto do circuito.
Além disso, sinais de GNSS são muito fracos, o que exige cuidados adicionais no projeto do circuito.

Toda a fundamentação teórica está baseada nos documentos exibidos na Tabela \ref{tab:referencias}.

\begin{table}[h]
    \centering
    \caption{Principais referências usadas no projeto.}
    \label{tab:referencias}
    \begin{tabular}{c|c}
    \textbf{Documento} & \textbf{URL} \\ \hline
    Manual do Hardware Teseo-LIV3f da ST & \href{https://drive.google.com/file/d/1f-_5UErfYuVgv-IKlNlQNE1Cbe42LFDT/view?usp=drive_link}{Disponível em: $<$ Link $>$} \\
    Manual do Software do Teseo-LIV3f da ST & \href{https://drive.google.com/file/d/19D5KxDNutAY975VEr7zWNG4IR6LbrYQc/view?usp=drive_link}{Disponível em: $<$ Link $>$} \\
    White Paper de boas práticas para layout de circuitos de RF & \href{https://drive.google.com/file/d/17EAXY1Gev_J_ZbgDPay-ZY9WCtNQap2y/view?usp=drive_link}{Disponível em: $<$ Link $>$} \\
    Guia da ST para circuitos RF & \href{https://drive.google.com/file/d/15d90DjYcKHO8I-ajw5f05wGQfgVGjobQ/view?usp=drive_link}{Disponível em: $<$ Link $>$} \\
    Tese de análise de antenas de patch cerâmica & \href{https://drive.google.com/file/d/1AVxLt7DhbOgQ1vpqmCdnyk6QVYBsWJfR/view?usp=drive_link}{Disponível em: $<$ Link $>$} \\ \hline
    \end{tabular}
\end{table}

Um White Paper é um documento publicado por uma empresa que aborda os detalhes técnicos de como 
um determinado produto é desenvolvido, sendo assim uma excelente fonte de informações 
para a equipe.

\section{Antenas Patch Cerâmicas}

A antena patch cerâmica é um tipo de antena de microfita compacta, formada por um elemento metálico (patch) sobre um substrato cerâmico de alta permissividade dielétrica e um plano de terra na face oposta.
A estrutura básica está exibida na Figura \ref{fig:antena_ceramica}.

\begin{figure}[h]
    \centering
    \includegraphics[width=0.6\textwidth]{images/antena_ceramica.png} 
    \caption{Antena Patch Cerâmica.}
    \label{fig:antena_ceramica}
\end{figure}

A partir dos valores da estrutura física da antena, ela apresenta uma determinada resposta em frequência, 
de modo que ela consegue capturar e transmitir sinais em uma faixa de frequência específica.
No caso do GNSS, a antena deve ser projetada para operar na faixa de 1.5 GHz, que é a frequência
em que os sinais de satélites GNSS são transmitidos.

A Figura \ref{fig:resposta_antena} mostra a resposta em frequência de uma antena patch cerâmica típica para GNSS.
Note que ela opera apenas na faixa de interesse de 1.5 GHz em que os satélites 
enviam os sinais. Além disso, esse comportamento evita interferência de outros sinais 
em frequências diferentes, como o LoRa que opera em 900 MHz.

\begin{figure}[h]
    \centering
    \includegraphics[width=0.75\textwidth]{images/resposta_antena.png} 
    \caption{Resposta em frequência de uma antena patch cerâmica para GNSS.}
    \label{fig:resposta_antena}
\end{figure}

Antenas de patch cerâmicas podem ainda ser classificadas como ativas ou passivas:

\begin{itemize}
    \item Ativas: apresentam amplificador LNA embutido na antena, de modo que o sinal 
    enviado à placa já chega amplificado, e basta enviá-lo a um módulo GNSS (no nosso, caso, o 
    Teseo-LIV3f) através de um filtro pi (será explicado mais a frente).
    \item Passivas: não possuem amplificador embutido, de modo que o sinal deve ser 
    amplificado na própria placa antes de chegar ao módulo GNSS.
\end{itemize}

O amplificador LNA comumente usado é o MAX2659 \url{https://www.digikey.com.br/pt/products/detail/analog-devices-inc-maxim-integrated/MAX2659ELT-T/2062078}.
No seguinte link uma antena de patch cerâmica ativa é desmontada, exibindo o circuito interno 
que amplifica o sinal \url{https://www.youtube.com/watch?v=s-jFprdDcM4}. 

\section{Circuitos RF}

Condutores que transportam sinais de alta frequência (RF) devem tomar uma série de cuidados especiais, 
uma vez que nessa frequência capacitâncias e indutâncias parasitas podem afetar significativamente 
o desempenho do circuito.

\subsection{Impedância Característica de Trilhas}

A impedância caracterísitica de uma trilha é dada pela Equação (\ref{eq:impedancia_trilha}):

\begin{equation} \label{eq:impedancia_trilha}
    Z_0 = \frac{87}{\sqrt{\epsilon_r + 1.41}} \ln \left( \frac{5.98h}{0.8w + t} \right) 
\end{equation}

\noindent onde:

\begin{itemize}
    \item $Z_0$: Impedância característica da trilha (Ohms)
    \item $\epsilon_r$: Permissividade relativa do material da PCB
    \item $h$: Espessura do dielétrico (distância entre a trilha e o plano de terra) (mm)
    \item $w$: Largura da trilha (mm)
    \item $t$: Espessura da trilha (mm)
\end{itemize}

Para circuitos GNSS, é ideal que as trilhas de RF possuam 
$Z_0 = 50 \, \Omega$. Como exemplo, considera uma trilha padrão usada pela equipe com o 
fabricante que nos patrocina. Temos os seguintes valores das constantes:

\begin{itemize}
    \item $\epsilon_r = 4.5$: varia entre 3.5 e 5.5, então assumimos a média
    \item $h = 1.5 \un{mm}$: é altura da PCB, pois o fabricante só nos fornece PCBs de duas layers.
    Se tivesse PCB de quatro layers, poderia ser $h = 0.508 \un{mm}$
    \item $w = 0.254 \un{mm}$: largura de trilha default, podemos ajustar no EasyEDA
    \item $t = 0.35 \mu \un{m}$: valor padrão
\end{itemize}

\noindent o que resulta em $Z_0 = 129.78 \, \Omega$.

É por esse motivo que PCBs de quatro layers são recomendadas: a distância entre 
as trilhas de RF e o plano de terra $h$ abaixo é menor, de modo que é mais fácil obter 
$Z_0 = 50 \, \Omega$.

Além disso, para que não exista mudanças na impedância da trilha, é recomendado que 
os pads dos componentes de RF sejam do mesmo tamanho da trilha, evitando descontinuidades
na largura da trilha. Para isso, use 0402 ou 0603 para que a largura dos pads e trilha seja  
a mais próxima possível.

\subsection{Plano de Terra}

Como visto na Equação (\ref{eq:impedancia_trilha}), a impedância característica de uma trilha depende
do plano de terra abaixo. Portanto, é essencial que o plano de terra seja contínuo,
sem cortes, fendas ou trilhas passando imediatamente por baixo das trilhas de RF,
para que a impedância característica não varie ao longo da trilha. 

Outro ponto essencial são as conexões do plano de terra. Devem ser usadas várias
vias conectando o plano de terra superior ao inferior, de modo que a impedância 
entre os planos seja minimizada. 

O plano de terra na camada das trilhas de RF também não pode estar muito próximo 
das trilhas de RF. Uma boa regra é manter uma distância entre o plano de GND e a trilha 
no mínimo igual à largura da trilha.


