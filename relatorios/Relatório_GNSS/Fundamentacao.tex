\chapter{Fundamentação Teórica}

Sinais GNSS têm frequência em torno de 1,5 GHz e são captados por antenas. Devido à alta frequência, é
necessário considerar boas práticas de projeto de circuitos de RF (radiofrequência) para garantir o
funcionamento correto do sistema. Além disso, os sinais GNSS são muito fracos, o que exige cuidados
adicionais no projeto da placa.

Toda a fundamentação teórica está baseada nos documentos listados na Tabela \ref{tab:referencias}.

\begin{table}[h]
    \centering
    \caption{Principais referências usadas no projeto.}
    \label{tab:referencias}
    \begin{tabular}{c|c}
    \textbf{Documento} & \textbf{URL} \\ \hline
    Manual do Hardware Teseo-LIV3f da ST & \href{https://drive.google.com/file/d/1f-_5UErfYuVgv-IKlNlQNE1Cbe42LFDT/view?usp=drive_link}{Disponível em: $<$ Link $>$} \\
    Manual do Software do Teseo-LIV3f da ST & \href{https://drive.google.com/file/d/19D5KxDNutAY975VEr7zWNG4IR6LbrYQc/view?usp=drive_link}{Disponível em: $<$ Link $>$} \\
    White Paper de boas práticas para layout de circuitos de RF & \href{https://drive.google.com/file/d/17EAXY1Gev_J_ZbgDPay-ZY9WCtNQap2y/view?usp=drive_link}{Disponível em: $<$ Link $>$} \\
    Guia da ST para circuitos RF & \href{https://drive.google.com/file/d/15d90DjYcKHO8I-ajw5f05wGQfgVGjobQ/view?usp=drive_link}{Disponível em: $<$ Link $>$} \\
    Tese de análise de antenas de patch cerâmica & \href{https://drive.google.com/file/d/1AVxLt7DhbOgQ1vpqmCdnyk6QVYBsWJfR/view?usp=drive_link}{Disponível em: $<$ Link $>$} \\ \hline
    \end{tabular}
\end{table}

Um white paper é um documento técnico publicado por uma empresa que descreve detalhes de projeto
e de implementação de um produto, sendo, portanto, uma fonte útil de informações para a equipe.

\section{Antenas Patch Cerâmicas}

A antena patch cerâmica é um tipo de antena de microfita compacta, formada por um elemento metálico (patch) sobre
um substrato cerâmico de alta permitividade dielétrica e um plano de terra na face oposta. A estrutura básica está
exibida na Figura \ref{fig:antena_ceramica}.

\begin{figure}[h]
    \centering
    \includegraphics[width=0.6\textwidth]{images/antena_ceramica.png} 
    \caption{Antena Patch Cerâmica.}
    \label{fig:antena_ceramica}
\end{figure}

A partir dos parâmetros físicos da antena obtém-se a resposta em frequência, que determina a faixa na qual a
antena é eficiente. No caso do GNSS, a antena deve ser projetada para operar na faixa de 1,5 GHz, frequência na qual
os sinais dos satélites GNSS são transmitidos.

A Figura \ref{fig:resposta_antena} mostra a resposta em frequência de uma antena patch cerâmica típica para GNSS.
Note que ela opera principalmente na faixa de interesse de 1,5 GHz. Esse comportamento reduz a
recepção de sinais em outras bandas, evitando interferências de fontes como o LoRa, que opera em 900 MHz.

\begin{figure}[h]
    \centering
    \includegraphics[width=0.75\textwidth]{images/resposta_antena.png} 
    \caption{Resposta em frequência de uma antena patch cerâmica para GNSS.}
    \label{fig:resposta_antena}
\end{figure}

Antenas de patch cerâmica podem ser classificadas como ativas ou passivas:

\begin{itemize}
    \item Ativas: apresentam amplificador LNA embutido na antena, de modo que o sinal
    que chega à placa já se apresenta amplificado; basta encaminhá-lo ao módulo GNSS (no nosso caso, o
    Teseo-LIV3f) através de um filtro pi (será explicado mais adiante).
    \item Passivas: não possuem amplificador embutido; o sinal deve ser amplificado na placa antes de chegar
    ao módulo GNSS.
\end{itemize}

O amplificador LNA comumente usado é o MAX2659 \url{https://www.digikey.com.br/pt/products/detail/analog-devices-inc-maxim-integrated/MAX2659ELT-T/2062078}.
No link a seguir, uma antena de patch cerâmica ativa é desmontada, mostrando o circuito interno que amplifica o sinal:
\url{https://www.youtube.com/watch?v=s-jFprdDcM4}.

\section{Circuitos RF}

Condutores que transportam sinais de alta frequência (RF) exigem cuidados específicos, pois, nessas
frequências, capacitâncias e indutâncias parasitas podem afetar significativamente o desempenho do circuito.

\subsection{Impedância característica de trilhas}

No caso de uma PCB com duas camadas e planos de GND em ambas camadas, as linhas 
de transmissão de RF se comportam como Coplanar Waveguide (CPW), exibida 
na Figura \ref{fig:cpw}.

\begin{figure}[h]
    \centering
    \includegraphics[width=0.5\textwidth]{images/cpw.png} 
    \caption{Coplanar Waveguide (CPW).}
    \label{fig:cpw}
\end{figure}

Na Figura \ref{fig:cpw}, temos  

\begin{itemize}
    \item W: distância entre a trilha de RF e os planos de GND adjacentes 
    \item S: largura da trlha de RF
    \item h: distância entre a trilha e o plano de GND na camada inferior
    \item $\epsilon_r$: constante dielétrica do material da PCB. Normalmente usamos FR-4 como 
    dielétrico, que possui $\epsilon_r = 4.5$
\end{itemize}

Para circuitos GNSS, é desejável que as trilhas de RF apresentem $Z_0 = 50 \, \Omega$.
Como exemplo, consideremos uma trilha padrão usada pela equipe, fornecida pelo fabricante que nos patrocina.
Temos os seguintes valores adotados:

\begin{itemize}
    \item $\epsilon_r = 4.5$ por usar FR-4
    \item $h = 1.5\,\un{mm}$: altura do dielétrico para placas de duas camadas (fornecidas pelo fabricante).
    Se a placa fosse de quatro camadas, com uma camada de GND imediatamente abaixo da camada onde está as trilhas de RF ,
    poderia ser $h = 0.508\,\un{mm}$.
    \item $W = S = 0.254\,\un{mm}$: largura de trilha padrão; pode ser ajustada no roteador (EasyEDA).
\end{itemize}

\noindent Isso resulta em $Z_0 = 86,9 \, \Omega$ usando uma calculadora de impedância 
disponível em \url{https://chemandy.com/calculators/coplanar-waveguide-with-ground-calculator.htm}.
A expressão que calcula a impedância de uma CPW vem de integrais elípticas, sendo mais fácil só usar 
uma calculadora online.

Por esse motivo, placas de circuito impresso (PCBs) de quatro camadas são recomendadas: a distância
$h$ entre as trilhas de RF e o plano de terra é menor, facilitando a obtenção de $Z_0 = 50 \, \Omega$.

Além disso, para evitar mudanças na impedância da trilha, recomenda-se que os pads dos componentes de RF
tenham largura compatível com a das trilhas, evitando descontinuidades. Prefira tamanhos como 0402 ou 0603,
para que a largura dos pads e das trilhas seja o mais próxima possível.

\subsection{Plano de terra}

Como visto na Equação (\ref{eq:impedancia_trilha}), a impedância característica de uma trilha depende
do plano de terra situado abaixo dela. Portanto, é essencial que o plano de terra seja contínuo,
sem cortes, fendas ou trilhas passando imediatamente por baixo das trilhas de RF, para que a
impedância característica não varie ao longo do trajeto.

Outro ponto importante são as conexões do plano de terra: recomenda-se o uso de múltiplas vias que
conectem o plano superior ao inferior, de modo a reduzir a indutância e a impedância entre as camadas.

O plano de terra na camada das trilhas de RF também não deve ficar muito distante nem muito próximo
às trilhas de RF. Uma regra prática é manter a distância entre o plano de terra (GND) e a trilha, no
mínimo, igual à largura da trilha.


