\chapter{Projeto}

A Figura \ref{fig:blocos_pcb} mostra um diagrama de blocos com os principais componentes
do projeto e sua função básica.

\begin{figure}[h]
    \centering
    \includegraphics[width=\textwidth]{images/blocos_pcb.png} 
    \caption{Diagrama de blocos dos principais componentes da placa.}
    \label{fig:blocos_pcb}
\end{figure}

Em essência, a antena capta o sinal GNSS, que é filtrado e enviado ao módulo Teseo. O Teseo processa o sinal
e envia os dados de posição ao microcontrolador STM32U5 por UART, usando o protocolo NMEA. O STM32U5 pode então
encaminhar esses dados via CAN para o restante do carro através do MCP2515 conectado ao barramento CAN.

\section{Esquemático}

A Figura \ref{fig:aquisicao_rf} mostra o circuito de aquisição do sinal da antena.
O sinal entra pelo conector H6, um conector IPEX padrão para antenas GNSS. Em seguida, ele 
passa pelo filtro LC composto por L3 e C12, cujos valores estão especificados no manual do hardware da ST.
O sinal VCC\_RF vem do Teseo; RF\_IN é o sinal de RF que entra no Teseo para ser processado pelo módulo GNSS.

\begin{figure}[h]
    \centering
    \includegraphics[width=0.8\textwidth]{images/aquisicao_rf.png} 
    \caption{Circuito de aquisição do sinal da antena.}
    \label{fig:aquisicao_rf}
\end{figure}

Observe que a antena é ativa e, portanto, necessita de alimentação. Tanto a alimentação quanto o sinal
passam pelo mesmo pino CONT (2) em H6. O capacitor C11 bloqueia a componente DC do sinal,
de modo que apenas o sinal RF é enviado ao Teseo. O indutor L3 bloqueia a componente RF,
fazendo com que apenas a componente DC seja enviada à antena para alimentá-la.

A Figura \ref{fig:teseo_circuito} mostra o circuito do Teseo projetado; novamente, ele segue quase à risca o que está
prescrito no manual do hardware.

\begin{figure}[h]
    \centering
    \includegraphics[width=\textwidth]{images/teseo_circuito.png} 
    \caption{Circuito do Teseo projetado.}
    \label{fig:teseo_circuito}
\end{figure}

As principais interfaces da Figura \ref{fig:teseo_circuito} estão listadas abaixo:

\begin{itemize}
    \item VCC\_RF: alimentação do sinal RF.
    \item RF\_IN: sinal RF que entra no Teseo.
    \item SDA\_GNSS: linha de dados I2C entre o STM e o Teseo.
    \item SCL\_GNSS: linha de clock I2C entre o STM e o Teseo.
    \item RX/TX\_GNSS: linhas UART entre o STM e o Teseo.
    \item PPS: pulso com período de 1 segundo emitido pelo módulo; pode ser usado para indicar, por exemplo, que o
    Teseo está operando.
    \item WAKEUP\_GNSS: pino usado para acordar o módulo de um estado de baixo consumo (ativo em 5 V).
    \item RESET\_GNSS: pino para resetar o Teseo (ativo em 0 V).
\end{itemize}

Note que não são necessários resistores pull-up no I2C, uma vez que eles já
estão presentes internamente no Teseo. Os filtros de 56 pF estão especificados
no manual.

A alimentação do Teseo usa o 3.3 V da placa, passando por um filtro pi exibido na 
Figura \ref{fig:filtro_pi}, especificado no manual do hardware da ST.

\begin{figure}[h]
    \centering
    \includegraphics[width=0.5\textwidth]{images/filtro_pi.png} 
    \caption{Filtro para alimentação do Teseo.}
    \label{fig:filtro_pi}
\end{figure}

Os demais circuitos da placa — alimentação GLV, comunicação CAN e circuitos
auxiliares do STM (capacitores de desacoplamento, botões de reset/boot, programação)
— são semelhantes aos das demais placas da equipe, em particular o IMU, que também usa o
STM32U5. Assim, esses circuitos não serão detalhados aqui.

\section{PCB}

\subsection{Circuitos RF}

A Figura \ref{fig:layout_rf} (a) mostra o layout das trilhas de RF usadas no projeto,
comparado com o layout recomendado pela ST na Figura \ref{fig:layout_rf} (b).

\begin{figure}[h]
    \centering
    \includegraphics[width=0.8\textwidth]{images/layout_rf.png} 
    \caption{Layout das trilhas de RF na placa (a) implementado e (b) recomendado pela ST.}
    \label{fig:layout_rf}
\end{figure}

Algumas observações importantes sobre a Figura \ref{fig:layout_rf}:

\begin{itemize}
    \item Os componentes das trilhas de RF estão em encapsulamento 0603, para terem largura compatível com a trilha.
    \item L3 está posicionado em T com a trilha, conforme sugerido no manual.
    \item A largura das trilhas de RF é 0,762 mm (30 mils). Isso resulta em
    $Z_0 = 94,2 \, \Omega$ para uma PCB de duas camadas (nosso caso) e
    $Z_0 = 55,47 \, \Omega$ para quatro camadas com 0,508 mm entre as camadas externas e
    intermediárias (cenário ideal, que não é o nosso caso, pois o patrocinador só imprime PCBs de
    duas camadas).
    \item Não há trilhas passando por baixo do Teseo, conforme recomendado.
    \item O posicionamento das múltiplas vias segue o layout recomendado.
\end{itemize}

A Figura \ref{fig:layout_rf_gnd} mostra o layout com ambos planos de GND exibidos.
Mais observações importantes sobre a Figura \ref{fig:layout_rf_gnd} estão listadas
abaixo:

\begin{figure}[h]
    \centering
    \includegraphics[width=0.8\textwidth]{images/layout_rf_gnd.png} 
    \caption{Layout das trilhas de RF na placa com os planos de GND exibidos.}
    \label{fig:layout_rf_gnd}
\end{figure}

\begin{itemize}
    \item Não há trilhas passando embaixo do Teseo e embaixo dos circuitos de RF, conforme recomendado 
    \item O plano de GND na camada inferior cobre toda a área abaixo do circuito de RF de forma contínua.
    \item A distância entre as trilhas RF e o plano de terra na camada superior é 0,508 mm.
    \item A isolação no pino 1 do Teseo (canto inferior direito) é feita com apenas
    uma via (o manual indica que pode ser uma ou duas); foi adicionada manualmente uma região
    sólida de keep-out após a criação do plano de GND no EasyEDA.
\end{itemize}

Observe que as trilhas no canto direito do Teseo não são de alta frequência e,
portanto, possuem largura padrão de 0,254 mm e não precisam seguir as
diversas recomendações do manual.


\subsection{Demais Circuitos}

O posicionamento do restante do circuito é feito em duas camadas, ao contrário das
demais placas da equipe, que em geral posicionam componentes apenas na camada superior.
Componentes "altos" estão na camada superior para não atrapalharem o encapsulamento,
enquanto componentes pequenos e de menor relevância estão na camada inferior para minimizar o tamanho da placa.
Todos os conectores estão na camada superior, no canto direito, para facilitar o acesso dos chicotes
quando a placa estiver instalada no carro.

A Figura \ref{fig:top_bottom_layer} destaca os top e bottom layers da placa.

\begin{figure}[h]
    \centering
    \includegraphics[width=\textwidth]{images/top_bottom_layer.png} 
    \caption{Camada superior (vermelho) e camada inferior (azul) da placa.}
    \label{fig:top_bottom_layer}
\end{figure}

Em suma, na camada superior estão posicionados todos os conectores, o regulador 5 V do GLV, todo o circuito RF,
LEDs indicadores, a maior parte dos circuitos do STM32U5 e todos os jumpers. Na camada inferior estão posicionados
o regulador 3,3 V, parte do circuito da CAN, alguns capacitores de desacoplamento do STM32U5 e capacitores de filtro da alimentação.

As trilhas de alimentação possuem largura de 20 mils (0,508 mm), enquanto as demais trilhas de sinal (exceto as de RF)
possuem largura de 10 mils (0,254 mm). Note-se que a trilha de alimentação do Teseo tem 10 mils porque a corrente é baixa.

\section{Software}

\subsection{STM32U5}

O código executado pelo STM32U5 deve executar as seguintes funções.

\begin{enumerate}
    \item Salvar as strings NMEA recebidas do Teseo via UART em um buffer
    \item Processar as strings NMEA para extrair os dados de posição (latitude, longitude)
    \item Atualizar uma estrutura de dados (struct) global com os dados de posição 
    \item Enviar os dados struct através do periférico FDCAN para o restante do carro
\end{enumerate}

Com o FreeRTOS, uma tarefa pode processar as strings NMEA e atualizar a estrutura de dados global, enquanto outra
tarefa periodicamente lê essa estrutura atual e a escreve no CAN. É importante que a tarefa de processamento da UART opere
na mesma taxa de amostragem em que o Teseo está configurado.

As strings NMEA enviadas pelo Teseo têm o formato exibido na Figura \ref{fig:strings_nmea}.
Basta criar um código em C que as processe e extraia as variáveis de latitude e longitude, bem como verifique
se o fix é válido; caso contrário, a mensagem deve ser descartada.

\begin{figure}[h]
    \centering
    \includegraphics[width=0.75\textwidth]{images/strings_nmea.png} 
    \caption{Strings NMEA enviadas pelo Teseo via UART.}
    \label{fig:strings_nmea}
\end{figure}

\subsection{Configurações do Teseo-LIV3f}

O Teseo possui uma memória interna configurável onde é possível salvar diversos
parâmetros de operação, como explicado na seção 2.2.2 do manual do software.
Em particular, duas configurações são importantes para atender aos requisitos do projeto:

\begin{enumerate}
    \item Aumentar baud rate para 115200: \$PSTMSETPAR,3102,0x9
    \item Aumentar taxa de amostragem para 5 Hz: \$PSTMSETPAR,1303,0.2
\end{enumerate}

Os comandos acima devem ser enviados via UART para o Teseo. Eles ainda não foram testados.
Se o Teseo parar de funcionar após enviar esses comandos, é possível resetar os parâmetros
para os valores de fábrica com o comando: \$PSTMRESTOREPAR (não testado ainda).

\section{Posicionamento no Carro}

Devido aos componentes na camada inferior, o encapsulamento deve prever um ressalto de 5 mm
nos quatro cantos, onde são colocados os parafusos (ou fitas-trava).

Em relação ao posicionamento no carro, é planejado posicionar a placa no bico do carro, perto 
de onde atualmente fica a placa de freio. Assim, tanto a placa quanto a antena ficarão embaixo do bico, 
que segundo nossos estudos, não afetará o sinal de GNSS por ser de fibra de vidro.

A Figura \ref{fig:posicionamento} mostra a localização da placa no chassi.

\begin{figure}[h]
    \centering
    \includegraphics[width=0.75\textwidth]{images/posicionamento.png} 
    \caption{Planejamento inicial do posicionamento da placa no carro.}
    \label{fig:posicionamento}
\end{figure}

Dessa forma, os chicotes entram pelo canto esquerdo e a antena fica à direita na
Figura \ref{fig:posicionamento}. Ainda é necessário realizar testes em bancada para verificar
experimentalmente se a fibra de vidro do bico efetivamente não atenua o sinal GNSS.

\section{Bill of Materials}

A BOM do projeto está exibida na Tabela \ref{tab:componentes}.
O custo total previsto é de R\$ 97,00, sem contar os componentes patrocinados (STM32U5, Teseo e antenas)
e sem considerar os custos que serão diluídos na compra conjunta com outras placas.

\begin{table}[htbp]
\centering
\caption{Bill of Materials do projeto.}
\label{tab:componentes}
\begin{tabular}{llrr}
\toprule
\textbf{Nome} & \textbf{Especificação} & \textbf{Qtd.} & \textbf{Preço Total (\$)} \\ 
\midrule
LED & LED-RED & 6 & 0,006 \\
Choke & CHOKE-SRF4532-1 & 1 & 0,6 \\
Regulador 3V3 & LM1117MPX-3.3v & 1 & 0,4 \\
Regulador 5V & Module\_MP1584EN & 1 & 3,41 \\
Diodo & RS1M & 1 & 0,57 \\
Conversor CAN & MCP2515-I/SO & 1 & 1,57 \\
Fusível & 500 mA Fuse 2410 & 1 & 0,12 \\
Cristal & 8 MHz & 1 & 0,07 \\
BJT & MUN2214T1G & 1 & 0,03 \\
C0805 & 1uF & 5 & 0,046 \\
C0805 & 10uF & 1 & 0,01 \\
C0805 & 22uF & 1 & 0,02 \\
C0805 & 100nF & 11 & 0,044 \\
C0805 & 10nF & 1 & 0,006 \\
C0805 & 56pF & 4 & 0,02 \\
C0805 & 4.7uF & 1 & 0,02 \\
C0805 & 22pF & 2 & 0,02 \\
Cap Eletrolítico & 100uF / 35V & 1 & 0,06 \\
C0402 & 120pF & 1 & 0,002 \\
C0402 & 56pF & 1 & 0,001 \\
C0603 & 120pF & 1 & 0,004 \\
C0603 & 56pF & 1 & 0,003 \\
L0805 & 27nH & 1 & 0,135 \\
L0402 & 47nH & 1 & 0,05 \\
L0603 & 47nH & 1 & 0,04 \\
R0805 & 4k7 & 5 & 0,01 \\
R0805 & 60R & 2 & 0,004 \\
R0805 & 10k & 7 & 0,014 \\
R0805 & 5k1 & 2 & 0,004 \\
R0805 & 0R & 1 & 0,002 \\
Conector & TYPE-C 6PIN & 1 & 0,05 \\
Antena Conector & KH-IPEX & 3 & 0,21 \\
Botao SMD & K2-3.6$\times$6.1 & 2 & 0,14 \\
Jumper & PTH 2 vias & 5 & \\
Jumper & PTH 3 vias & 1 & \\
Header & PTH 2 vias & 2 & \\
Nanofit 2 x 1 90 graus & GLV & 1 & \\
Microfit 2 x 2 90 graus & CAN & 1 & \\
Nanofit 3 x 02 & PROGG & 1 & \\
Header & PTH 3 x 02 vias & 1 & \\
STM32U5 & & & (patrocínio) \\
Teseo-LIV3f & & & (patrocínio) \\
Antena & BT-580 & & 24,69 \\
Antena & BT-0001 & & 35,00 \\
\midrule
Total & & & R\$ 97,00 \\
\bottomrule
\end{tabular}
\end{table}

