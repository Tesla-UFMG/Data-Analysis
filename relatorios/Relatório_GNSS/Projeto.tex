\chapter{Projeto}

A Figura \ref{fig:blocos_pcb} mostra um diagram de blocos com os principais componentes 
do projeto e sua função básica.

\begin{figure}[h]
    \centering
    \includegraphics[width=\textwidth]{images/blocos_pcb.png} 
    \caption{Diagrama de blocos dos principais componentes da placa.}
    \label{fig:blocos_pcb}
\end{figure}

Em essência, a antena capta o sinal GNSS, que é filtrado e enviado ao módulo Teseo. O Teseo processa o sinal
e envia os dados de posição ao microcontrolador STM32U5 via UART usando o protocolo 
NMEA. O STM32U5 pode então enviar esses dados
via CAN para o restante do carro através do MCP2515 conectado ao barramento CAN.

\section{Esquemático}

A Figura \ref{fig:aquisicao_rf} mostra o circuito de aquisição do sinal da antena.
O sinal entra pelo conector H6, um conector IPEX padrão para antenas GNSS. Em seguida, ele 
passa pelo filtro LC composto por L3 e C12, cujos valores estão especificados no manual do hardware da ST.
O sinal VCC\_RF vem do Teseo, e RF\_IN é o que entra no Teseo para ser processado pelo módulo GNSS.

\begin{figure}[h]
    \centering
    \includegraphics[width=0.8\textwidth]{images/aquisicao_rf.png} 
    \caption{Circuito de aquisição do sinal da antena.}
    \label{fig:aquisicao_rf}
\end{figure}

Note que a antena é ativa, logo ela precisa de alimentação. Tanto a alimentação quanto o sinal
passam pelo mesmo pino CONT (2) em H6. O capacitor C11 bloqueia a componente DC do sinal,
de modo que apenas o sinal RF é enviado ao Teseo. O indutor L3 bloqueia a componente RF,
de modo que apenas a componente DC é enviada à antena para alimentá-la.

A Figura \ref{fig:teseo_circuito} mostra o circuito do Teseo projetado. Novamente ele segue quase à risca o que está 
no manual do hardware.

\begin{figure}[h]
    \centering
    \includegraphics[width=\textwidth]{images/teseo_circuito.png} 
    \caption{Circuito do Teseo projetado.}
    \label{fig:teseo_circuito}
\end{figure}

As principais interfaces da Figura \ref{fig:teseo_circuito} estão listadas abaixo:

\begin{itemize}
    \item VCC\_RF: alimentação do sinal RF.
    \item RF\_IN: sinal RF que entra no Teseo.
    \item SDA\_GNSS: linha de dados I2C entre o STM e o Teseo.
    \item SCL\_GNSS: linha de clock I2C entre o STM e o Teseo.
    \item RX / TX \_GNSS: linhas de UART entre o STM e o Teseo.
    \item PPS: onda quadrada com período de 1 segundo emitida pelo módulo. Usada para piscar um LED, indicando que o 
    Teseo está vivo.
    \item WAKEUP\_GNSS: pino usado para acordar o módulo de um estado de baixo consumo. (ativo em 5 V)
    \item RESET\_GNSS: pino para resetar o Teseo. (ativo em 0 V)
\end{itemize}

Note que não é necessário resistores pull-up no I2C uma vez que eles já 
estão presentes internamentes no Teseo. Os filtros de 56 pF estão especificados 
no manual. 

A alimentação do Teseo usa o 3.3 V da placa, passando por um filtro pi exibido na 
Figura \ref{fig:filtro_pi}, especificado no manual do hardware da ST.

\begin{figure}[h]
    \centering
    \includegraphics[width=0.5\textwidth]{images/filtro_pi.png} 
    \caption{Filtro para alimentação do Teseo.}
    \label{fig:filtro_pi}
\end{figure}

Os demais circuito da placa: alimentação GLV, comunicação CAN e circuitos 
auxiliares do STM (capacitores de acoplamento, botões de reset / boot, programação)
são iguais às demais placas da equipe, em particular o IMU que também usa o 
STM32U5. Assim, esses circuitos não serão detalhados aqui.

\section{PCB}

\subsection{Circuitos RF}

A Figura \ref{fig:layout_rf} (a) mostra o layout das trilhas de RF usadas no projeto,
comparada com o layout recomendado pela ST na Figura \ref{fig:layout_rf} (b).

\begin{figure}[h]
    \centering
    \includegraphics[width=0.8\textwidth]{images/layout_rf.png} 
    \caption{Layout das trilhas de RF na placa (a) implementado e (b) recomendado pela ST.}
    \label{fig:layout_rf}
\end{figure}

Algumas observações importantes sobre a Figura \ref{fig:layout_rf}:

\begin{itemize}
    \item Os componentes das trilhas de RF são 0603 para terem a mesma largura da trilha 
    \item L3 está posicionado em T com a trilha, como sugerido no manual 
    \item A largura das trilhas de RF é 0.762 mm (30 mils). Isso resulta em 
    $Z_0 = 94.2 \, \Omega$ para uma PCB de duas layers (nosso caso) e 
    $Z_0 = 55.47 \, \Omega$ para quatro layers com 0.508 mm entre as layers externas e
    intermediárias (cenário ideal, que não é o nosso caso pois o patrocinador só imprime PCBs de 
    duas layers)
    \item Note que não há trilhas passando embaixo do Teseo, conforme recomendado
    \item O posicionamento das múltiplas vias segue o layout recomendado
\end{itemize}

A Figura \ref{fig:layout_rf_gnd} mostra o layout com ambos planos de GND exibidos.
Mais observações importantes sobre a Figura \ref{fig:layout_rf_gnd} estão listadas
abaixo:

\begin{figure}[h]
    \centering
    \includegraphics[width=0.8\textwidth]{images/layout_rf_gnd.png} 
    \caption{Layout das trilhas de RF na placa com os planos de GND exibidos.}
    \label{fig:layout_rf_gnd}
\end{figure}

\begin{itemize}
    \item Não há trilhas passando embaixo do Teseo e embaixo dos circuitos de RF, conforme recomendado 
    \item O plano de GND no bottom layer cobre toda a área abaixo do circuito de RF de forma 
    contínua
    \item A distância entre as trilhas RF e o plano de terra no top layer é 0.508 mm 
    \item A "isolação" no pino 1 do Teseo no canto inferior direito é feita com somente 
    uma via (manual diz que pode ser feito com uma ou duas), manualmente adicionando uma região 
    sólida de keep-out depois que o plano de GND é feito no EasyEDA
\end{itemize}

Note que as trilhas no canto direito do Teseo não são de alta frequência, e 
portanto possuem largura default de 0.254 mm e não precisam seguir as 
diversas recomendações do manual.


\subsection{Demais Circuitos}

O posicionamento do restante do circuito é feito em duas layers, ao contrário das
demais placas da equipe que em geral posicionam componentes apenas no top layer. 
Componentes "altos" estão no top layer para não atrapalharem o encapsulamento,
enquanto componentes pequenos e de baixa relevância estão no bottom layer para minimizar o tamanho da placa.
Todos os conectores estão no top layer no canto direito para facilitar o acesso dos chicotes
quando a placa estiver no carro.

A Figura \ref{fig:top_bottom_layer} destaca os top e bottom layers da placa.

\begin{figure}[h]
    \centering
    \includegraphics[width=\textwidth]{images/top_bottom_layer.png} 
    \caption{Top (vermelho) e bottom (azul) layers da placa.}
    \label{fig:top_bottom_layer}
\end{figure}

Em suma, no top layer estão posicionados: todos os conectores,
regulador 5 V do GLV, todo o circuito RF, 
LEDs indicadores, maior parte dos circuitos do STM32U5, todos os jumpers. 
No bottom layer estão posicionados: regulador 3.3 V, parte do circuito da CAN,
alguns capacitores de acoplamento do STM32U5 e capacitores de filtro da alimentação.

Trilhas de alimentação possuem largura de 20 mils (0.508 mm), enquanto as demais 
trilhas de sinal (exceto as de RF) possuem largura de 10 mils (0.254 mm). Note que a alimentação 
do Teseo é de 10 mils pois é de baixa potência.

\section{Posicionamento e Encapsulamento}


\section{Software}

\subsection{STM32U5}

O código executado pelo STM32U5 deve executar as seguintes funções.

\begin{enumerate}
    \item Salvar as strings NMEA recebidas do Teseo via UART em um buffer
    \item Processar as strings NMEA para extrair os dados de posição (latitude, longitude)
    \item Atualizar uma estrutura de dados (struct) global com os dados de posição 
    \item Enviar os dados struct através do periférico FDCAN para o restante do carro
\end{enumerate}

Usando freeRTOS, é possível que uma thread processe as strings NMEA e atualize
a estrutura de dados global, enquanto outra thread sempre pega a estrutura de dados 
atual e a escreve no CAN. É importante que a thread de processamento da UART opera na mesma taxa de amostragem 
em que o Teseo está configurado. 

As strings NMEA enviadas pelo Teseo tem o formato exibido na Figura \ref{fig:strings_nmea}.
Basta criar um código em C que as processa e extrai as variáveis de latitude e longitude, 
bem como se o fix é válido - se o fix é inválido, deve-se descartar a mensagem.

\begin{figure}[h]
    \centering
    \includegraphics[width=0.75\textwidth]{images/strings_nmea.png} 
    \caption{Strings NMEA enviadas pelo Teseo via UART.}
    \label{fig:strings_nmea}
\end{figure}

\subsection{Configurações do Teseo-LIV3f}

O Teseo possui uma memória interna configurável onde é possível salvar diversas
parâmetros de operação como explicado na seção 2.2.2 do manual do software.
Em particular, duas configurações são importantes para atender aos requisitos do projeto 

\begin{enumerate}
    \item Aumentar baud rate para 115200: \$PSTMSETPAR,3102,0x9
    \item Aumentar taxa de amostragem para 5 Hz: \$PSTMSETPAR,1303,0.2
\end{enumerate}

Os comandos acima devem ser enviados via UART para o Teseo. Eles ainda não foram testados.
Se o Teseo parar de funcionar após enviar esses comandos, é possível resetar os parâmetros
para os valores de fábrica com o comando: \$PSTMRESTOREPAR (não testado ainda).



