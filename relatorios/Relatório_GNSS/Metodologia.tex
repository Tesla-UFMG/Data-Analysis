\chapter{Metodologia}

\section{Revisão da Literatura}
Inicialmente, foi desenvolvido um estudo sobre o projeto realizado na temporada de 2025, junto a uma pesquisa sobre o protocolo de comunicação CAN e conversas com ex-membro. Nessa fase, todas as fontes de pesquisa foram documentadas em um relatório bibliográfico.  

A partir disso, têm-se um entendimento dos componentes necessários para a montagem da placa e os desafios que serão enfrentados durante o seu desenvolvimento. Além disso, uma projeção financeira foi realizada a partir dos componentes apresentados no projeto anterior. 

\section{Especificação da Antena}

\section{Projeto do Circuito}

Em seguida, os circuitos do projeto serão desenvolvidos e simulados no software \textit{LTspice}, designado para simular circuitos eletrônicos. Nessa etapa, calcula-se uma alteração nos componentes e circuitos da IsoCAN 2025, e também, um entendimento da causa de não conclusão da placa. Essas mudanças serão documentadas em um relatório técnico.

\section{Design da PCB}

Logo após o dimensionamento, o esquemático dos circuitos será feito através da plataforma \textit{EasyEDA}, no qual também será realizado o design da placa de circuito impresso (PCB) com 2 camadas (layers). Nessa etapa, também será desenvolvido um relatório parcial da IsoCAN.
