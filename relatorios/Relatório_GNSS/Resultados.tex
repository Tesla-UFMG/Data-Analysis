\chapter{Resultados}

Esta seção descreve testes realizados com uma placa antiga de GNSS desenvolvida pela equipe em 2020,
que utiliza um circuito praticamente idêntico ao da placa projetada, mudando apenas o layout da PCB.
O objetivo dos testes é validar o funcionamento do circuito e obter dados de desempenho do GNSS.

\section{Precisão}

O teste foi realizado debaixo das árvores atrás do Bloco 4, ao lado do estacionamento da Eng, em frente ao Coltec.
De acordo com o Google Earth, a posição registrada foi:

\begin{itemize}
    \item Latitude: 19º 52' 16'' Sul = 19.87133 S
    \item Longitude: 43º 57' 46'' Oeste = 43.96309 W
    \item Altitude: 821 m
\end{itemize}

A antena está conectada à placa verde do GNSS desenvolvida antes de 2022
(Módulo GNSS 2.0 - EasyEDA $>$ Aquisição de Dados 2020 $>$ Módulo GNSS 2.0 $>$
PCB Placa GNSS 2.0), como mostra a imagem abaixo. Deixe-a com a parte rosa voltada para o céu.
A Figura \ref{fig:circuito_teste} exibe o circuito usado no teste.

\begin{figure}[h]
    \centering
    \includegraphics[width=\textwidth]{images/circuito_teste.png} 
    \caption{Circuito de teste da placa antiga da equipe.}
    \label{fig:circuito_teste}
\end{figure}

Mais detalhes de como reproduzir o teste encontram-se no drive da equipe:
\url{https://docs.google.com/document/d/1_DcsvIPrMl5WaGMswOeWJAnK461f2M6uIb7QwfO3SD4/edit?usp=sharing}.

Os resultados obtidos estão resumidos na Figura \ref{fig:resultado_teste}.
Observe que esse teste já atende ao requisito R2 de precisão do GNSS.

\begin{figure}[h]
    \centering
    \includegraphics[width=0.8\textwidth]{images/resultado_teste.png} 
    \caption{Resultado obtido no teste de precisão da placa antiga.}
    \label{fig:resultado_teste}
\end{figure}

Além disso, foi verificado que o circuito e a antena não conseguem captar os sinais dos satélites se:

\begin{itemize}
    \item estiverem dentro de uma sala ou oficina;
    \item estiverem perto de algum prédio alto.
\end{itemize}

Por esse motivo, o teste foi realizado nas árvores do Coltec, longe de quaisquer edificações e com boa visibilidade do céu.

\section{Interferência de Fibra de Vidro}

(completar)

\section{Comparação com a Núcleo GNSS da ST}

(completar - nova nucleo só chegou ontem 19/12/25)

\chapter{Conclusão}

Tendo em vista os objetivos do projeto, foi possível projetar uma placa para adquirir
as coordenadas geográficas do carro a partir de GNSS. O circuito foi concebido com base
em diversas fontes e manuais, além de seguir o esquema da placa antiga da equipe,
que já apresenta boa precisão em determinadas condições.

O layout proposto introduz melhorias em relação à placa antiga, com maior cuidado
quanto à impedância e ao roteamento das trilhas de RF.

Mais testes são necessários para aferir a interferência da fibra de vidro e para
testar efetivamente a placa completa quando a fabricação da placa for concluída e os componentes chegarem.