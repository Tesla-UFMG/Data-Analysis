\chapter{Resultados}

Essa seção aborda testes realizados com uma placa antiga de GNSS desenvolvida pela equipe em 2020, 
que utiliza um cirucito praticamente idêntico ao da placa projetado, mudando apenas o layout da PCB.
O objetivo dos testes é validar o funcionamento do circuito e obter dados de desempenho do GNSS.

\section{Precisão}

O teste foi realizado embaixo das árvores que ficam atrás do Bloco 4, do lado do estacionamento da Eng, em frente ao Coltec.
Segundo o Google Earth, estamos posicionados em:

\begin{itemize}
    \item Latitude: 19º 52' 16'' Sul = 19.87133 S
    \item Longitude: 43º 57' 46'' Oeste = 43.96309 W
    \item Altitude: 821 m
\end{itemize}

A antena está conectada à placa verde do GNSS desenvolvida antes de 2022 
(Módulo GNSS 2.0 - EasyEDA $>$ Aquisição de Dados 2020 $>$ Módulo GNSS 2.0 $>$ 
PCB Placa GNSS 2.0) como mostra a imagem abaixo. Deixe ela com essa parte 
rosa virada para o céu. A Figura \ref{fig:circuito_teste} 
exibe o circuito usado no teste.

\begin{figure}[h]
    \centering
    \includegraphics[width=\textwidth]{images/circuito_teste.png} 
    \caption{Circuito de teste da placa antiga da equipe.}
    \label{fig:circuito_teste}
\end{figure}

Mais detalhes de como reproduzir o teste se encontram no drive da equipe:
\url{https://docs.google.com/document/d/1_DcsvIPrMl5WaGMswOeWJAnK461f2M6uIb7QwfO3SD4/edit?usp=sharing}.

Os resultados obtidos estão sumarizados na Figura \ref{fig:resultado_teste}.
Note que esse teste já atende ao requisito R2 da precisão do GNSS.

\begin{figure}[h]
    \centering
    \includegraphics[width=0.8\textwidth]{images/resultado_teste.png} 
    \caption{Resultado obtido no teste de precisão da placa antiga.}
    \label{fig:resultado_teste}
\end{figure}

Além disso, foi verificado que o circuito e antena não conseguem 
captar os sinais dos satélites se:

\begin{itemize}
    \item Estiver dentro de uma sala ou oficina
    \item Perto de algum prédio alto
\end{itemize}

\noindent sendo esse o motivo pelo qual o 
teste foi feito nas árvores do Coltec, 
longe de qualquer prédio e com boa visibilidade para o céu.

\section{Interferência de Fibra de Vidro}

(completar)

\section{Comparação com a Núcleo GNSS da ST}

(completar - nova nucleo só chegou ontem 19/12/25)

\chapter{Conclusão}

Tendo em vista os objetivos do projeto, foi possível projar uma placa para aquisitar 
as coordenadas geográficas do carro a partir de GNSS. O circuito foi projetado 
com base em diversas fontes e manuais, além de seguir o circuito da placa antiga da equipe 
que já apresenta uma boa precisão em certas condições. 

Além disso, o layout da placa propõe uma série de melhorias em relação à placa antiga,
com um maior cuidado em relação à impedância das trilhas de RF e roteamento dessas trilhas.

Mais testes são necessários para afetir a interferência da fibra de vidro, 
e efetivamente testar a placa completa quando a impressão da placa for concluída 
e os componentes chegarem.